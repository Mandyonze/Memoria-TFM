%----------
%	RESUMEN Y PALABRAS CLAVE
%----------	

\setcounter{page}{5}
	

Directorio Activo, del inglés {\it Active Directory (AD)}, es el servicio de directorio proporcionado por Microsoft y que tiene como finalidad principal la gestión de manera eficiente y centralizada de la información y los recursos de una empresa. En la actualidad, Active Directory es utilizado por la mayoría de las organizaciones a nivel mundial y se considera una de las partes fundamentales para el correcto funcionamiento de una empresa. La información gestionada por Active Directory permite gestionar usuarios como pueden ser empleados, clientes,  proveedores y que éstos puedan localizar los dispositivos, recursos y servicios distribuidos por la red como pueden ser ordenadores, servidores, impresoras, bases de datos, etc. Como se puede deducir, debido a la importancia que tiene Active Directory dentro de una organización y la información que gestiona, le sitúan en uno de los principales objetivos para atacantes y ciberdelincuentes. Comprometer el servicio Active Directory supone un gran problema de seguridad para una empresa si el sistema principal de gestión se ve comprometido. Es por ello, que en los últimos años ha aumentado considerablemente el ataque a Active Directory cuya finalidad es hacerse con el control de la empresa y comprometer la seguridad de la información. Con el fin de contribuir al desarrollo, el trabajo realizado se centra en el análisis de las principales amenazas o ataques que pueden comprometer Active Directory gestionado por la última versión lanzada por Microsoft: Windows Server 2019. Esto se ha logrado mediante la creación de un laboratorio de pruebas, de manera local, que ha permitido la creación de una empresa ficticia y la revisión de una batería de los principales ataques analizados. \\

\textit{\textbf{Palabras Clave:}}

Microsoft Active Directory, Domain Controller, Kerberos, Ciberseguridad, Windows Server, Pentesting, Red Team
	
		
\vfill
\newpage %página en blanco o de cortesía
\thispagestyle{empty}
\mbox{}
