%----------
%	RESUMEN Y PALABRAS CLAVE
%----------	

\setcounter{page}{5}
	
Active Directory (AD) es una base de datos distribuida que permite a los administradores gestionar de manera eficiente la información de la empresa y los límites de la misma. En la actualidad, Active Directory es considerado una de las partes fundamentales para el correcto funcionamiento de cualquier empresa. La información gestionada por Active Directory, engloba usuarios, clientes, proveedores, dispositivos (ordenadores, servidores, impresoras...), servicios y aplicaciones así como la interacción entre ellos. Por lo tanto, supone uno de los principales objetivos para atacantes y ciberdelincuentes así como una gran problema de seguridad para una organización si el sistema de gestión principal se ve comprometido. En los últimos años se ha producido un gran aumento de ataques contra Active Directory cuyo objetivo es hacerse con el control de la empresa y comprometer la seguridad de la información, esto es debido a que se han perfeccionado los principales ataques así como la implementación de nuevas herramientas que facilitan todo tipo de ataques. \\

Con el fin de contribuir al desarrollo y mejora de la seguridad de Active Directory y por ende, de las empresas y particulares que hacen uso de ello, el trabajo realizado se centra en el estudio de los principales componentes que engloban la seguridad y gestión del sistema de directorio Active Directory a través de la creación de un laboratorio, de manera local, y el análisis y revisión de los principales ataques y amenazas usadas en la actualidad para vulnerar dicho sistema de gestión. Para ello, se implementa una topología que simula una empresa ficticia e implementa la última versión de Windows Server 2019.\\ 

\textit{\textbf{Palabras Clave:}}

Microsoft Active Directory, Domain Controller, Kerberos, Ciberseguridad, Windows Server, Pentesting, Red Team
	
		
\vfill
\newpage %página en blanco o de cortesía
\thispagestyle{empty}
\mbox{}
