%----------
%	RESUMEN Y PALABRAS CLAVE
%----------	

\setcounter{page}{5}
	
Microsoft Active Directory se ha convertido en una de las partes fundamentales de las empresas en la actualidad y se le considera como el {\it core} de la organización. Active Directory permite a los administradores gestionar de manera eficiente la información de la empresa y los límites de la misma. Esta información puede englobar usuarios, clientes, proveedores, dispositivos como ordenadores, servidores o impresoras, servicios y aplicaciones y la forma de interacción entre ellos así como los diferentes permisos que tiene cada usuario o grupo de usuarios. Esta información, como se puede deducir, supone un importante objetivo para atacantes y ciberdelincuentes y un gran problema de seguridad para las empresas si estos sistemas de gestión se ven comprometidos. Esto a llevado a cabo un gran aumento de ciberataques contra Active Directory con el objetivo principal de recolectar dicha información y comprometer la seguridad total de una empresa. Además, en los últimos años se han perfeccionado las técnica principales de ataque así como la implementación de novedosas herramientas que facilitan todo tipo de ataques.\\ 

Con el fin de contribuir al desarrollo y mejora de la seguridad de Active Directory y por ende, de las empresas y particulares que hacen uso de ello, el trabajo realizado se centra en el estudio de los principales componentes que engloban la seguridad y gestión del sistema de directorio Active Directory a través de la creación de un laboratorio, de manera local, y el análisis y revisión de los principales ataques y amenazas usadas en la actualidad para vulnerar dicho sistema de gestión. Además, se ha implementado una topología que simula una empresa ficticia e implementa las últimas versiones proporcionadas por Microsoft que gestionan cada dominio. \\

\textit{\textbf{Palabras Clave:}}

Microsoft Active Directory, Domain Controller, Kerberos, Ciberseguridad, Windows Server, Pentesting, Red Team
	
		
\vfill
\newpage %página en blanco o de cortesía
\thispagestyle{empty}
\mbox{}
