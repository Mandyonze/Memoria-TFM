En este capítulo se van a discutir los resultados obtenidos tanto en la creación del laboratorio como la experimetnación de los principales ataques que afectan a Active Directory en la última versión de Windows Server 2019.

\subsubsection{Creación del laboratorio}
El laboratorio propuesto para la implentación de las pruebas realizadas consta de 4 máquinas: en primer lugar DC01 representa el Domain Controller que gestiona Active Directory formada por un Windows Server 2019, en segundo lugar la máquina Cliente01 representa a un usuario o empleado (o varios usuarios) unidos al dominio cuya máquina puede ser comprometida, en tercer lugar la máquina Gateway que sirve de enlace entre la red interna y la red externa que representa internet, por último, la máquina denominada Atacante01 representa un atacante, miembro de Red Team o auditor de seguridad que compromete un sistema del dominio y tiene acceso a la red interna. Se ha comprobado que con estas cuatro máquinas es posible replicar los ataques planteados en el Capítulo de experimentación. 
\subsubsection{Pass the hash}
En cuanto al ataque de Pass The Hash, se ha comprobado que es posible replicarlo en entornos actualizados, la premisa para este ataque es comprometer un sistema de dominio con suficientes privilegios como para obtener las credenciales almacenadas en el equipo de un cliente. El enfoque más efectivo para la protección de un equipo frente a este tipo de ataques es implementar políticas de seguridad para que hashes de cuentas privilegiadas como puede ser de {\it Domain Admin} no se almacenen en memoria o sea imposible extraer dichos hashes.
\subsubsection{NTLM Relay}
Para evitar ataques de NTLM Relay, Microsoft ha implementado varias medidas, entre ellas y la más efectiva ha sido incluida en el parche MS08-068 que imposibilita ataques de NTLM Reflejado, es decir, que no es posible transmitir el Hash NTLM a la misma máquina que se obtuvo, aunque como se ha visto en la literatura es posible enviarlo a otro servicio o recurso. Por otro lado, la otra medida es la firma de los paquetes SMB para evitar que el mensaje se altere y proteger la integridad del mensaje. Esta medida imposibilita los ataques de SMB Relay basados en el protocolo SMB aunque esta medida no está activada por defecto en la mayoría de los equipos con sistema operativo Windows. Pese a estas medidas, se ha comprobado la eficacia de este ataque ya que la única premisa es que el atacante esté en la misma red que el dominio y ha sido posible realizar un volcado de la base de datos SAM. 
\subsubsection{Overpass the hash}
Este ataque ha sido posible realizarlo del mismo modo que el ataque {\it pass the hash}. {\it Overpass the hash} permite realizar estos ataques en el paquete de autenticación Kerberos, ampliamente utilizado para la autenticación en servicios y recursos utilizados por Active Directory.
\subsubsection{Pass the ticket}
La eficacia del ataque {\it pass the ticket} reside en la obtención de un Ticket TGT válido que resulte de interés para acceder a recursos importantes como puede ser un Domain Controller o equipos con información confidencial. A diferencia de ataques de {\it pass the hash u overpass the hash} no es necesario tener una sesión de administrador local para obtener los Tickets TGS algo que puede ser de utilidad cuando no se ha conseguido elevar privilegios en una máquina comprometida.
\subsubsection{Golden Ticket}
La técnica {\it Golden Ticket} te permite mantener persistencia una vez comprometido un Domain Controller de manera eficaz y transparente a los administradores de sistemas. Es necesario disponer del hash de la contraseña de administración {\it krbtgt} lo dificulta la realización de este ataque. Una vez obtenida, se ha comprobado que es posible obtener tickets TGT suplantando a cualquier usuario del dominio algo que puede ser crítico para organizaciones y empresas. 
\subsubsection{Kerberoast}
Por último, el ataque Kerberoast es el menos ``ruidoso'' ya que únicamente se realizan peticiones legítimas al dominio y una vez obtenidos los tickets TGS se puede realizar la fase de cracking en una máquina local ajena al dominio. Pese a ello, este ataque reside su eficacia en credenciales débiles de los servicios y recursos lo que dificulta así su efectividad. 
