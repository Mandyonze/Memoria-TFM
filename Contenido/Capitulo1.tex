En los últimos años, con el desarrollo intrínseco de la tecnología y el aumento masivo de información, empresas y organizaciones a nivel mundial se han visto en la necesidad de disponer de sistemas y/o servicios que les permitan administrar de una manera lógica, estructurada y eficaz tanto los usuarios como la información y recursos distribuidos en la red de los que disponen para el correcto funcionamiento del negocio. Microsoft Active Directory~\cite{Capitulo1:Microsoft} se presenta como una solución efectiva a esta problemática. Estos recursos engloban bases de datos, sistemas de ficheros, aplicaciones web, servidores, impresoras, etc. Además, Active Directory sirve para gestionar la autenticación y autorización de dichos recursos, es decir, permite administrar qué usuarios pueden, o no, acceder a dichos recursos.\\

En la actualidad, Active Directory es la solución elegida por más del 90\% de las empresas y organizaciones a nivel mundial~\cite{Capitulo1:Percent} para la gestión y administración de los recursos e información de una empresa. Esto supone que la amplia mayoría de atacantes elijan Active Directory como el objetivo, o {\it target}, principal en el ciclo de vida de un ataque dirigido a una organización. La finalidad principal es comprometer la infraestructura, obtener información confidencial o realizar ataques de Ransomware para extorsionar o sacar beneficio económico. Esto se puede observar en la manera de atacar de los principales grupos de ciberdelincuentes o grupos organizados como puede ser APT28, APT29, Cobal Strike, etc. o en los últimos ataques de Ransomware como WannaCry, NotPetya, MBR-ONI~\cite{Capitulo1:Ransomware}~\cite{Capitulo1:Ransomware2}~\cite{Capitulo1:Ransomware3} que ponen los servicios de Active Directory en el punto de mira y centro de sus ataques. Además, la aparición de vulnerabilidades críticas y el desarrollo de herramientas cada vez más sofisticadas posibilitan considerablemente la explotación afectando a la seguridad de Active Directory. \\

Estas características hacen que Active Directory sea un importante objeto de estudio para investigadores y equipos tanto de {\it Red Team} como {\it Blue Team}. El trabajo realizado ha abordado este problema y presenta como objetivo principal la revisión, análisis y prueba de alguna de las principales amenazas que ponen en grave riesgo la seguridad de Active Directory en la última versión de Windows Server.\\

\section{Estado del Arte}

Active Directory fue lanzado por primera vez en 1999 con el Sistema Operativo {\it Windows 2000 Server Edition}. Desde entonces, proteger, mantener actualizado y crear una infraestructura sólida y segura ha sido uno de los principales objetivos de Microsoft. Para ello, se ha instaurado una política de actualizaciones semestrales con plazo de servicio de 18 meses~\cite{Capitulo1:Update} que corrige las vulnerabilidades encontradas y propone nuevas implementaciones que mejoren tanto el uso como la seguridad.\\

Para el desarrollo de este proyecto se ha utilizado la última versión disponible: Windows Server 2019 1903 como Domain Controller para la gestión y administración de Active Directory. Windows Server 2019 está basado en la versión más estable y optimizada de Windows Server 2016 y se han añadido mejoras considerables que se pueden consultar en~\cite{Capitulo1:WindowsServer2019} y que destacan, en términos de seguridad, la implementación de un sofisticado antivirus para la protección de amenzas: {\it Windows Defender Advanced Threat Protection (ATP)}, un nuevo conjunto de funciones para la identificación y prevención de intrusiones: {\it Windows Defender ATP Exploit Guard} y novedades en la seguridad con {\it Software Defined Networking (SDN)} introducido en versiones anteriores. \\

Por otro lado, en cuanto a la experimentación, se ha utilizado una topología de Active Directory que permite evaluar satisfactoriamente los ataques analizados. Aunque existen multitud de ataques diferentes y variaciones de los mismos, se ha considerado los siguientes ataques para delimitar el límite del proyecto realizado: 

\begin{itemize}
\item \textbf{Pass-The-Hash}
\item \textbf{NTLM Relay}
\item \textbf{Overpass-The-Hash}
\item \textbf{Pass-The-Ticket}
\item \textbf{Golden Ticket}
\item \textbf{Kerberoast}
\end{itemize}

La gran mayoría de estos ataques sobre Active Directory, no son específicos de este sino que aprovechan debilidades en los protocolos de autenticación utilizados. En la actualidad, los protocolos utilizados principalmente son: Microsoft NTLM y Kerberos Version 5 Protocol. Por este motivo, ambos protocolos se van a detallar y analizar en los capítulos posteriores. \\

\section{Objetivos}

Como se ha comentado anteriomente, el objetivo principal de este trabajo es la revisión y análisis de las principales amenazas que pueden comprometer la seguridad de Active Directory. Para ello, es necesario la recreación de un laboratorio de pruebas que permita replicar dichos ataques.\\

Como se ha comentado anteriormente, debido a la importacia del caso de estudio, este proyecto tiene como objetivo la adquisición de conocimiento sobre la infraestructura Active Directory como punto de partida tanto para otro tipo de investigaciones dejándolas así como trabajo futuro además de la aplicación de la topología con nuevos Domain Controllers, servidores, etc. Por otro lado, el conocimiento de los principales ataques y cómo está organizado es de gran importacia hoy en día, permitiendo así una correcta implementación que minimice los riesgos a los que está sometida una organización. \\

Por último, este trabajo tiene como objetivo establecer las pautas y directrices para la creación de un laboratorio que permita tanto a {\it Pentesters} o profesionales de la seguridad ofensiva para realizar ejercicios simulados de {\it Red Team} en entornos controlados como a administradores de sistemas o equipos de {\it Blue Team} para probar nuevas configuraciones o realizar simulaciones de actualizaciones o mejoras en un entorno simulado. 

\section{Organización del Proyecto}

El presente documento se divide en 6 capítulos, en los cuales en primera instancia se detallan los aspectos a tener en cuenta relacionados con Active Directoy, se detalla el laboratorio implementado, las pruebas que se van a realizar, la experimentación realizada así como los resultados obtenidos durante el trascurso:\\

En el Capítulo 2 se ha realizado un análisis de la autenticación y autorización que se lleva a cabo en los sistemas operativos Windows, definiendo así el inicio de sesión que realiza un usuario legítimo así como el proceso de autorización del recurso al que intenta acceder. \\

En el Capítulo 3 se han definido en profundidad los principales protocolos de au\-ten\-ti\-ca\-ción: MSV1\_0 y Kerberos siendo estos los más utilizados en la au\-ten\-ti\-ca\-ción en Sistemas Windows. \\

En el Capítulo 4 se introduce la terminología y los conceptos clave que engloba Active Directory para posteriormente definir las directrices para la creación del laboratorio de pruebas. \\

En el Capítulo 5 se analizan los principales ataques sobre Active Directory listados anteriormente. \\

En el Capítulo 6 se presentan y discuten los resultados obtenidos tras la ex\-pe\-ri\-men\-ta\-ción de los diferentes ataques y técnicas. \\

En el Capítulo 7, para finalizar, se enumeran las conclusiones obtenidas tras la re\-a\-li\-za\-ción de este proyecto además de proponer posibles líneas futuras de este trabajo. \\



