
\section{Conclusiones}

A lo largo del trabajo se han revisado en profundidad las diferentes técnicas y ataques que puedes comprometer la seguridad de un Active Directory en la última versión proporcionada por Micorosft: Windows Server 2019. Se ha implementado un laboratorio local que permite la replicación de dichos ataques y diferentes pruebas en un entorno controlado sin afectar a una implementación de una empresa u organización. Esto ha requerido la instalción de difernetes sistemas operativos, la configuración de una topología de red que puede simular a un entorno real, la configuración de Active Directory y la creación de diferentes usuarios con distintos privilegios. Una vez montado el laboratorio de pruebas ha sido posible replicar dichos ataques de manera satisfactoria. \\

En cuanto a la experimentación realizada, todos los ataques elegidos se han podido replicar de manera satisfactoria en la última versión de Windows Server, por lo que se puede concluir que, aún siguiendo una política de actualizaciones es posible que se pueda comprometer un servicio de directorio como Active Directory de una empresa. En las últimas actualizaciones, Microsoft, ha intentado mitigar estos ataques o al menos reducir su impacto implemtando medidas como las vistas anteriormente. En técnicas como Pass the hash, es posible que no sea una vulnerabilidad como tal sino una implmentación al modelo de {\it Singl Sign-On (SSO)} que permite al usuario acceder a recursos y servicios sin introducir la contraseña cada vez que se requiera autenticación. \\

La realización de este proyecto ha servido para adquirir una base en Active Directory y conocer los principales ataques utilizados por atacantes y profesionales de la seguridad informática. El conocimiento adquirido es de gran importancia hoy en día debido a la multitud de empresas que utilizan Active Directory como servicio de directorio para gestionar los recursos en red.\\

\section{Trabajo Futuro}

A lo largo de este proyecto se han encontrado limitaciones que han sido asumidas y definidas fuera del alcance de este proyecto, como puede ser la fase de enumeración, acceso inicial, explotación y elevación de privilegios en Sistemas Windows centrando este proyecto en movimientos laterales y movimientos verticales a través de una red. Por lo que se define como objeto de estudio futuro las fases previas a la compromisión de un sistema. \\

En cuanto al laboratorio, este trabajo se ha centrado en un único dominio {\it Laboratory.com} como objeto de estudio. Las empresas de hoy en día disponen de multitud de forest, dominios y subdominios debido a la gran cantidad de recurosso a gestionar, por lo que sería interesante ampliar el laboratorio con distintos forest y dominios y envestigar cómo gestiona Windows las relaciones de confianza entre ellos. \\

Por último, en cuanto a los ataque sería de gran utilidad probar diferentes variaciones a dichos ataques o la experimentación de otros ataques interesantes como {\it DCSync}. Otra de las limitaciones asumidas a lo largo del desarrollo del trabajo ha sido la utilización de herramientas como Mimikatz. Estas herramientas son objeto de estudio por los antivirus y es importante que el ataque no sea detectado por administradores de sistemas o el equipo de {\it Blue Team}. Por lo tanto, como trabaja futuro sería la investigación de dichas herramientas y la detección que hacen activirus como Windows Defender.
