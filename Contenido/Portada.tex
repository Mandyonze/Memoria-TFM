\thispagestyle{empty}


%----------
%	PORTADA
%----------	
\begin{titlepage}
	\begin{sffamily}
	\color{azulUC3M}
	\begin{center}
		\begin{figure}[H] %incluimos el logotipo de la Universidad
			\makebox[\textwidth][c]{\includegraphics[width=16cm]{Portada_Logo.png}}
		\end{figure}
		\vspace{1cm}
		\begin{Large}
			Máster Universitario en Ciberseguridad\\			
			2018-2019\\
			\vspace{1cm}		
			\textsl{Trabajo Fin de Máster}
			\bigskip
			
		\end{Large}
		 	% {\Huge ``CIBEREJERCICIOS PARA EVALUAR ACTIVE DIRECTORY EN SUS DISTINTAS VERSIONES. ''}\\
		 	{\Huge ``Ciberejercicios para evaluar \\
		 	Active Directory en sus distinas versiones. ''}\\
		 	\vspace*{0.5cm}
	 		\rule{10.5cm}{0.1mm}\\
			\vspace*{0.9cm}
			{\LARGE Borja Lorenzo Fernádez}\\ 
			\vspace*{1cm}
		\begin{Large}
			Tutor\\
			Andrés Marín López\\
		\end{Large}
	\end{center}
	
	\vfill
	\color{black}
	\begin{footnotesize}
	\noindent\fbox{
	\begin{minipage}{\textwidth}
		\textbf{DETECCIÓN DEL PLAGIO}\\
		La Universidad utiliza el programa \textbf{Turnitin Feedback Studio} para comparar la originalidad del trabajo entregado por cada estudiante con millones de recursos electrónicos y detecta aquellas partes del texto copiadas y pegadas. Copiar o plagiar en un TFM es considerado una \textbf{\underline{Falta Grave}}, y puede conllevar la expulsión definitiva de la Universidad.
	\end{minipage}	
	}
	% \vspace*{.5cm}\\	
	% \noindent\includegraphics[width=4.2cm]{imagenes/creativecommons.png}\\
	% \emph{[Incluir en el caso del interés en su publicación en el archivo abierto]}\\
	% Esta obra se encuentra sujeta a la licencia Creative Commons \textbf{Reconocimiento - No Comercial - Sin Obra Derivada}

	\end{footnotesize}
	\end{sffamily}
\end{titlepage}

\newpage %página en blanco o de cortesía
\thispagestyle{empty}
\mbox{}