%----------
%   IMPORTANTE
%----------

% Si nunca has utilizado LaTeX es conveniente que aprendas una serie de conceptos básicos antes de utilizar esta plantilla. Te aconsejamos que leas previamente algún tutorial (puedes encontar muchos en Internet).

% Esta plantilla está basada en las recomendaciones de la guía "Trabajo fin de Máster: Escribir el TFM", que encontrarás en http://uc3m.libguides.com/TFM/escribir
% contiene recomendaciones de la Biblioteca basadas principalmente en estilos APA e IEEE, pero debes seguir siempre las orientaciones de tu Tutor de TFM y la normativa de TFM para tu titulación.

% Encontrarás un ejemplo de TFM realizado con esta misma plantilla en la carpeta "_ejemplo_TFM_2019". Consúltalo porque contiene ejemplos útiles para incorporar tablas, figuras, listados de código, bibliografía, etc.



%----------
%	CONFIGURACIÓN DEL DOCUMENTO
%----------

% Definimos las características del documento y añadimos una serie de paquetes (\usepackage{package}) que agregan funcionalidades a LaTeX.

\documentclass[12pt]{report} %fuente a 12pt

%------------------------------------------------------------------------------------


\hyphenation{ciber-delincuentes pro-por-cionadas com-pro-meter ges-tio-nar la-bo-ra-torio net-working pro-por-cio-na-do cen-tra-li-za-da im-ple-men-ta-do he-rra-mien-tas in-fra-es-truc-tu-ra Di-rec-to-ry im-ple-men-ta-cio-nes auten-ti-ca-cion u-ti-li-za-dos} 
\PassOptionsToPackage{hyphens}{url}
\usepackage{float}
\UseRawInputEncoding

\usepackage{cite}


\usepackage{color}
\definecolor{gray97}{gray}{.97}
\definecolor{gray75}{gray}{.75}
\definecolor{gray45}{gray}{.45}

\usepackage{listings}
\lstset{ frame=Ltb,
framerule=0pt,
aboveskip=0.5cm,
framextopmargin=3pt,
framexbottommargin=3pt,
framexleftmargin=0.4cm,
framesep=0pt,
rulesep=.4pt,
backgroundcolor=\color{gray97},
rulesepcolor=\color{black},
%
stringstyle=\ttfamily,
showstringspaces = false,
basicstyle=\small\ttfamily,
commentstyle=\color{gray45},
keywordstyle=\bfseries,
%
numbers=left,
numbersep=15pt,
numberstyle=\tiny,
numberfirstline = false,
breaklines=true,
}

% minimizar fragmentado de listados
\lstnewenvironment{listing}[1][]
{\lstset{#1}\pagebreak[0]}{\pagebreak[0]}

\lstdefinestyle{consola}
{basicstyle=\scriptsize\bf\ttfamily,
backgroundcolor=\color{gray75},
}

\lstdefinestyle{C}
{language=C,
}

%------------------------------------------------------------------------------------



% MÁRGENES: 2,5 cm sup. e inf.; 3 cm izdo. y dcho.
\usepackage[
a4paper,
vmargin=2.5cm,
hmargin=3cm
]{geometry}

% INTERLINEADO: Estrecho (6 ptos./interlineado 1,15) o Moderado (6 ptos./interlineado 1,5)
\renewcommand{\baselinestretch}{1.15}
\parskip=6pt

% DEFINICIÓN DE COLORES para portada y listados de código
\usepackage[table]{xcolor}
\definecolor{azulUC3M}{RGB}{0,0,102}
\definecolor{gray97}{gray}{.97}
\definecolor{gray75}{gray}{.75}
\definecolor{gray45}{gray}{.45}

% Soporte para GENERAR PDF/A --es importante de cara a su inclusión en e-Archivo porque es el formato óptimo de preservación y a la generación de metadatos, tal y como se describe en http://uc3m.libguides.com/ld.php?content_id=31389625. En la carpeta incluímos el archivo plantilla_tfg_2017.xmpdata en el que puedes incluir los metadatos que se incorporarán al archivo PDF cuando lo compiles. Ese archivo debe llamarse igual que tu archivo .tex. Puedes ver un ejemplo en esta misma carpeta.
\usepackage[a-1b]{pdfx}

% ENLACES
\usepackage{hyperref}
\hypersetup{colorlinks=true,
	linkcolor=black, % enlaces a partes del documento (p.e. índice) en color negro
	urlcolor=blue} % enlaces a recursos fuera del documento en azul

% EXPRESIONES MATEMATICAS
\usepackage{amsmath,amssymb,amsfonts,amsthm}

\usepackage{txfonts} 
\usepackage[T1]{fontenc}
\usepackage[utf8]{inputenc}

\usepackage[spanish, es-tabla]{babel} 
\usepackage[babel, spanish=spanish]{csquotes}
\AtBeginEnvironment{quote}{\small}

% diseño de PIE DE PÁGINA
\usepackage{fancyhdr}
\pagestyle{fancy}
\fancyhf{}
\renewcommand{\headrulewidth}{0pt}
\rfoot{\thepage}
\fancypagestyle{plain}{\pagestyle{fancy}}

% DISEÑO DE LOS TÍTULOS de las partes del trabajo (capítulos y epígrafes o subcapítulos)
\usepackage{titlesec}
\usepackage{titletoc}

\titleformat{\chapter}[block]
{\large\bfseries} %\filcenter}
{\thechapter.}
{5pt}
{}
{}
\titlespacing{\chapter}{0pt}{0pt}{*3}
\titlecontents{chapter}
[0pt]                                               
{}
{\contentsmargin{0pt}\thecontentslabel.\enspace}
{\contentsmargin{0pt}}                        
{\titlerule*[.7pc]{.}\contentspage}                 

\titleformat{\section}
{\bfseries}
{\thesection.}
{5pt}
{}
\titlecontents{section}
[5pt]                                               
{}
{\contentsmargin{0pt}\thecontentslabel.\enspace}
{\contentsmargin{0pt}}
{\titlerule*[.7pc]{.}\contentspage}

\titleformat{\subsection}
{\normalsize\bfseries}
{\thesubsection.}
{5pt}
{}
\titlecontents{subsection}
[10pt]                                               
{}
{\contentsmargin{0pt}                          
	\thecontentslabel.\enspace}
{\contentsmargin{0pt}}                        
{\titlerule*[.7pc]{.}\contentspage}  


% DISEÑO DE TABLAS. Puedes elegir entre el estilo para ingeniería o para ciencias sociales y humanidades. Por defecto, está activado el estilo de ingeniería. Si deseas utilizar el otro, comenta las líneas del diseño de ingeniería y descomenta las del diseño de ciencias sociales y humanidades
\usepackage{multirow} % permite combinar celdas 
\usepackage{caption} % para personalizar el título de tablas y figuras
% \usepackage{floatrow} % utilizamos este paquete y sus macros \ttabbox y \ffigbox para alinear los nombres de tablas y figuras de acuerdo con el estilo definido. Para su uso ver archivo de ejemplo 
\usepackage{array} % con este paquete podemos definir en la siguiente línea un nuevo tipo de columna para tablas: ancho personalizado y contenido centrado
\newcolumntype{P}[1]{>{\centering\arraybackslash}p{#1}}
\DeclareCaptionFormat{upper}{#1#2\uppercase{#3}\par}

% Diseño de tabla para ingeniería
\captionsetup[table]{
	format=upper,
	name=TABLA,
	justification=centering,
	labelsep=period,
	width=.75\linewidth,
	labelfont=small,
	font=small,
}

%Diseño de tabla para ciencias sociales y humanidades
%\captionsetup[table]{
%	justification=raggedright,
%	labelsep=period,
%	labelfont=small,
%	singlelinecheck=false,
%	font={small,bf}
%}


% DISEÑO DE FIGURAS. Puedes elegir entre el estilo para ingeniería o para ciencias sociales y humanidades. Por defecto, está activado el estilo de ingeniería. Si deseas utilizar el otro, comenta las líneas del diseño de ingeniería y descomenta las del diseño de ciencias sociales y humanidades
\usepackage{graphicx}
\graphicspath{{imagenes/}} %ruta a la carpeta de imágenes

% Diseño de figuras para ingeniería
\captionsetup[figure]{
	format=hang,
	name=Fig.,
	singlelinecheck=off,
	labelsep=period,
	labelfont=small,
	font=small		
}

% Diseño de figuras para ciencias sociales y humanidades
%\captionsetup[figure]{
%	format=hang,
%	name=Figura,
%	singlelinecheck=off,
%	labelsep=period,
%	labelfont=small,
%	font=small		
%}


% NOTAS A PIE DE PÁGINA
\usepackage{chngcntr} %para numeración contínua de las notas al pie
\counterwithout{footnote}{chapter}

% LISTADOS DE CÓDIGO
% soporte y estilo para listados de código. Más información en https://es.wikibooks.org/wiki/Manual_de_LaTeX/Listados_de_código/Listados_con_listings
\usepackage{listings}

% definimos un estilo de listings
\lstdefinestyle{estilo}{ frame=Ltb,
	framerule=0pt,
	aboveskip=0.5cm,
	framextopmargin=3pt,
	framexbottommargin=3pt,
	framexleftmargin=0.4cm,
	framesep=0pt,
	rulesep=.4pt,
	backgroundcolor=\color{gray97},
	rulesepcolor=\color{black},
	%
	basicstyle=\ttfamily\footnotesize,
	keywordstyle=\bfseries,
	stringstyle=\ttfamily,
	showstringspaces = false,
	commentstyle=\color{gray45},     
	%
	numbers=left,
	numbersep=15pt,
	numberstyle=\tiny,
	numberfirstline = false,
	breaklines=true,
	xleftmargin=\parindent
}

\captionsetup[lstlisting]{font=small, labelsep=period}
% fijamos el estilo a utilizar 
\lstset{style=estilo}
\renewcommand{\lstlistingname}{\uppercase{Código}}


%BIBLIOGRAFÍA - PUEDES ELEGIR ENTRE ESTILO IEEE O APA. POR DEFECTO ESTÁ CONFIGURADO IEEE. SI DESEAS USAR APA, COMENTA LAS LÍNEA DE IEEE Y DESCOMENTA LAS DE APA. Si haces cambios en la configuración de la bibliografía y no obtienes los resultados esperados, es recomendable limpiar los archivos auxiliares y volver a compilar en este orden: COMPILAR-BIBLIOGRAFIA-COMPILAR

% Tienes más información sobre cómo generar bibliografía y CONFIGURAR TU EDITOR DE TEXTO para compilar con biber en http://tex.stackexchange.com/questions/154751/biblatex-with-biber-configuring-my-editor-to-avoid-undefined-citations , https://www.overleaf.com/learn/latex/Bibliography_management_in_LaTeX y en http://www.ctan.org/tex-archive/macros/latex/exptl/biblatex-contrib
% También te recomendamos consultar la guía temática de la Biblioteca sobre citas bibliográficas: http://uc3m.libguides.com/guias_tematicas/citas_bibliograficas/inicio

% CONFIGURACIÓN PARA LA BIBLIOGRAFÍA IEEE
% \usepackage[backend=biber, style=ieee, isbn=false,sortcites, maxbibnames=5, minbibnames=1]{biblatex} % Configuración para el estilo de citas de IEEE, recomendado para el área de ingeniería. "maxbibnames" indica que a partir de 5 autores trunque la lista en el primero (minbibnames) y añada "et al." tal y como se utiliza en el estilo IEEE.

%CONFIGURACIÓN PARA LA BIBLIOGRAFÍA APA
%\usepackage[style=apa, backend=biber, natbib=true, hyperref=true, uniquelist=false, sortcites]{biblatex}
%\DeclareLanguageMapping{spanish}{spanish-apa}

% Añadimos las siguientes indicaciones para mejorar la adaptación de los estilos en español
% \DefineBibliographyStrings{spanish}{%
% 	andothers = {et\addabbrvspace al\adddot}
% }
% \DefineBibliographyStrings{spanish}{
% 	url = {\adddot\space[En línea]\adddot\space Disponible en:}
% }
% \DefineBibliographyStrings{spanish}{
% 	urlseen = {Acceso:}
% }
% \DefineBibliographyStrings{spanish}{
% 	pages = {pp\adddot},
% 	page = {p.\adddot}
% }

% \addbibresource{Bibliografia/Bibliografia.bib} % llama al archivo bibliografia.bib en el que debería estar la bibliografía utilizada


%-------------
%	DOCUMENTO
%-------------

\begin{document}
\pagenumbering{roman} % Se utilizan cifras romanas en la numeración de las páginas previas al cuerpo del trabajo
	

%----------
%	PORTADA
%----------	

\thispagestyle{empty}


%----------
%	PORTADA
%----------	
\begin{titlepage}
	\begin{sffamily}
	\color{azulUC3M}
	\begin{center}
		\begin{figure}[H] %incluimos el logotipo de la Universidad
			\makebox[\textwidth][c]{\includegraphics[width=16cm]{Portada_Logo.png}}
		\end{figure}
		\vspace{1cm}
		\begin{Large}
			Máster Universitario en Ciberseguridad\\			
			2018-2019\\
			\vspace{1cm}		
			\textsl{Trabajo Fin de Máster}
			\bigskip
			
		\end{Large}
		 	% {\Huge ``CIBEREJERCICIOS PARA EVALUAR ACTIVE DIRECTORY EN SUS DISTINTAS VERSIONES. ''}\\
		 	{\Huge ``Ciberejercicios para evaluar \\
		 	Active Directory en sus distinas versiones. ''}\\
		 	\vspace*{0.5cm}
	 		\rule{10.5cm}{0.1mm}\\
			\vspace*{0.9cm}
			{\LARGE Borja Lorenzo Fernádez}\\ 
			\vspace*{1cm}
		\begin{Large}
			Tutor\\
			Andrés Marín López\\
		\end{Large}
	\end{center}
	
	\vfill
	\color{black}
	\begin{footnotesize}
	\noindent\fbox{
	\begin{minipage}{\textwidth}
		\textbf{DETECCIÓN DEL PLAGIO}\\
		La Universidad utiliza el programa \textbf{Turnitin Feedback Studio} para comparar la originalidad del trabajo entregado por cada estudiante con millones de recursos electrónicos y detecta aquellas partes del texto copiadas y pegadas. Copiar o plagiar en un TFM es considerado una \textbf{\underline{Falta Grave}}, y puede conllevar la expulsión definitiva de la Universidad.
	\end{minipage}	
	}
	% \vspace*{.5cm}\\	
	% \noindent\includegraphics[width=4.2cm]{imagenes/creativecommons.png}\\
	% \emph{[Incluir en el caso del interés en su publicación en el archivo abierto]}\\
	% Esta obra se encuentra sujeta a la licencia Creative Commons \textbf{Reconocimiento - No Comercial - Sin Obra Derivada}

	\end{footnotesize}
	\end{sffamily}
\end{titlepage}

\newpage %página en blanco o de cortesía
\thispagestyle{empty}
\mbox{}

%----------
%	Agracedimientos
%----------	
\chapter*{Agradecimientos}
%----------
%	Agracedimientos
%----------	

\setcounter{page}{3}
	
	% ESCRIBIR LA DEDICATORIA AQUÍ	
		
\vfill
\newpage %página en blanco o de cortesía
\thispagestyle{empty}
\mbox{}

%----------
%	RESUMEN Y PALABRAS CLAVE
%----------	
\chapter*{Resumen}
%----------
%	RESUMEN Y PALABRAS CLAVE
%----------	

\setcounter{page}{5}
	

Directorio Activo, del inglés {\it Active Directory (AD)}, es el servicio de directorio proporcionado por Microsoft y que tiene como finalidad principal la gestión de manera eficiente y centralizada de la información y los recursos de una empresa. En la actualidad, Active Directory es utilizado por la mayoría de las organizaciones a nivel mundial y se considera una de las partes fundamentales para el correcto funcionamiento de una empresa. La información gestionada por Active Directory permite gestionar usuarios como pueden ser empleados, clientes,  proveedores y que éstos puedan localizar los dispositivos, recursos y servicios distribuidos por la red como pueden ser ordenadores, servidores, impresoras, bases de datos, etc. Como se puede deducir, debido a la importancia que tiene Active Directory dentro de una organización y la información que gestiona, le sitúan en uno de los principales objetivos para atacantes y ciberdelincuentes. Comprometer el servicio Active Directory supone un gran problema de seguridad para una empresa si el sistema principal de gestión se ve comprometido. Es por ello, que en los últimos años ha aumentado considerablemente el ataque a Active Directory cuya finalidad es hacerse con el control de la empresa y comprometer la seguridad de la información. Con el fin de contribuir al desarrollo, el trabajo realizado se centra en el análisis de las principales amenazas o ataques que pueden comprometer Active Directory gestionado por la última versión lanzada por Microsoft: Windows Server 2019. Esto se ha logrado mediante la creación de un laboratorio de pruebas, de manera local, que ha permitido la creación de una empresa ficticia y la revisión de una batería de los principales ataques analizados. \\

\textit{\textbf{Palabras Clave:}}

Microsoft Active Directory, Domain Controller, Kerberos, Ciberseguridad, Windows Server, Pentesting, Red Team
	
		
\vfill
\newpage %página en blanco o de cortesía
\thispagestyle{empty}
\mbox{}






	

%-------------------------------------------------------------------------------
%	ÍNDICES
%-------------------------------------------------------------------------------

%--
%Índice general
%-
\tableofcontents
\thispagestyle{fancy}

\newpage %página en blanco o de cortesía
\thispagestyle{empty}
\mbox{}

%--
%Índice de figuras. Si no se incluyen, comenta las líneas siguientes
%-
\listoffigures
\thispagestyle{fancy}

\newpage %página en blanco o de cortesía
\thispagestyle{empty}
\mbox{}

%--
%Índice de tablas. Si no se incluyen, comenta las líneas siguientes
%-
\listoftables
\thispagestyle{fancy}

\newpage %página en blanco o de cortesía
\thispagestyle{empty}
\mbox{}


%-------------------------------------------------------------------------------
%-------------------------------------------------------------------------------

%----------
%	TRABAJO
%----------	
\clearpage
\pagenumbering{arabic} % numeración con múmeros arábigos para el resto de la publicación	


\chapter{Introducción}
En los últimos años, empresas y organizaciones se han visto en la necesidad de gestionar de una manera eficiente y centralizada la información y recursos en red que disponen, activo fundamental para el correcto funcionamiento del negocio. El aumento masivo de dicha información además de la necesidad de crear, distribuir y manipular tal cantidad de datos, ya sea a través de servicios de bases de datos como puede ser servicios MySQL, la obtención de servidores y dispositivos para su almacenamiento, la creación de  aplicaciones web y servicios que permitan su distribución o la gestión de los usuarios que puedan manipularla o consultarla supone una gran cantidad de agentes que intervienen en el funcionamiento que es necesario controlar y regular. Microsoft Active Directory supone una solución a esa problemática a través de un servicio de directorio como base de datos distribuida que permite la gestión, administración y localización de todos los recursos en red ~\cite{Capitulo1:Microsoft}.\\

Active Directory es el directorio principal para autenticación y autorización utilizado por el 90\% de las empresa a nivel mundial~\cite{Capitulo1:Percent}, es uno de los pilares fundamentales que sostienen los recursos en red de la mayoría de las empresas vigentes así como uno de los principales objetivos para atacantes  debido a dicha importancia. Esto se puede obsrevar en el principal interés que tienen los principales grupos de ciberdelincuentes como APT28, APT29, Cobalt Strike...  por conseguir acceso a la infraestructura Active Directory de una empresa o los ataques de ransomware WannaCry, NotPetya, MBR-ONI, etc, que ponen a Active Directory en el punto de mira y centro de sus ataques ~\cite{Capitulo1:Ransomware}~\cite{Capitulo1:Ransomware2}~\cite{Capitulo1:Ransomware3}. En los últimos años, se ha visto un aumento considerable de vulnerabilidades críticas que afectan a la seguridad y con ello la aparición de sofisticadas herramientas que detectan estas vulnerabilidades y hacen considerablemente más sencillo su explotación. \\

Por todo ello y con el fin de abordar esta problemática, el trabajo realizado se ha centrado en la revisión, análisis y prueba en profundidad de las principales amenazas que suponen un problema de seguridad para Active Directory en su última versión, técnicas como Pass-The-Hass, NTLM Relay, Kerberoast, etc que serán detalladas en los capítulos posteriores. Además, también se proporciona las directrices para la creación de un laboratorio local que permite la prueba de los ataques detallados así como la posibilidad de probar nuevas técnicas y ataques sin poner en riesgo la seguridad de ningún entorno real u organización.\\


\section{Estado del Arte}

Como se ha comentado anteriormente, la importancia de Active Directory viene de la mano de  un aumento considerable de amenazas. Proteger y mantener actualizado es una de los principales objetivos de Microsoft así como de los propios usuarios. Para ello, Microsoft ha instaurado un modelo de actualizaciones semestrales con plazo de servicio de 18 meses~\cite{Capitulo1:Update}.

Para este proyecto se ha utilizado la última versión disponible: Windows Server 2019 1903 como Domain Controller para la gestión y administración de Active Directory. Windows Server 2019 está basado en la versión más estable y optimizada de Windows Server 2016 y se han añadido mejoras considerables que se pueden consultar en~\cite{Capitulo1:WindowsServer2019} y que destacan, en  términos de seguridad, la implementación de un sofisticado antivirus para la protección de amenzas: {\it Windows Defender Advanced Threat Protection (ATP).}, un nuevo conjunto de funciones para la identificación y prevención de intrusiones: {\it Windows Defender ATP Exploit Guard} y novedades en la seguridad con {\it Software Defined Networking (SDN)} introducido en versiones anteriores. 

Por otro lado, en cuanto a la experimentación, para analizar la seguridad de una topología de Active Directory se han considerado los siquietes ataques:

\begin{itemize}
\item \textbf{Pass-The-Hash}
\item \textbf{NTLM Relay}
\item \textbf{Overpass-The-Hash}
\item \textbf{Pass-The-Ticket}
\item \textbf{Golden/Silver Ticket}
\item \textbf{Kerberoast}
\end{itemize}


Como se puede observar, la mayoría de los ataques no son específicos de Active Directory si no que atacan a los protocolos NTLM y Kerberos, por lo que, también se van a detallar en profundidad estos protocolos en los capitulos siguientes.


\section{Objetivos}

El objetivo principal de este trabajo es la creación de un laboratorio que provea la infraestructura necesaria para la replicación de las principales técnicas de ataque a Active Directory así como la revisión de las mismas sobre las diferentes versiones proporcionadas por Microsoft. La creación del laboratorio posibilita tanto a {\it Pentesters} o expertos en seguridad ofensiva realizar ejercicios de {\it Read Team} en entornos controlados o réplicas de un entorno real como a administradores de sistemas para probar nuevas configuraciones y reglas para equipos de {\it Blue Team}.\\

Además, este trabajo tiene como objetivo la adquisición de conocimiento sobre Active Directory como punto de partida para futuras investigaciones. Como se ha visto en la introducción Active Directory es una parte fundamental de una empresa y es uno de los principales objetivos de atacantes, conocer los principales ataques y cómo está organizado es de gran importancia hoy en día, permitiendo aasí una correcta implementación que minimice los riesgos a los que está sometido.\\


\section{Organización del Proyecto}

El presente documento se divide en 7 capítulos, en los cuales en primera instancia se detallan los aspectos a tener en cuenta relacionados con Active Directoy, se detalla el laboratorio implementado, las pruebas que se van a realizar, la experimentación realizada así como los resultados obtenidos durante el trascurso:\\

En el Capítulo 2 se detalla los aspectos relacionados con Active Directory, en primer lugar se define la autenticación y autorización en sistemas Windows tanto localmente como en dominio, se analizan los protocolos NT Lan Manager y Kerberos y la terminología relacionada con Active Directory.\\

En el Capitulo 3 se desarrolla la topología que se ha elegido para la creación del laboratorio de pruebas con las diferentes versiones de Windows y su implementación.\\

En el Capítulo 4 se detallan los ataques elegidos para realizar la experimentación.\\

En el Capítulo 5, una vez detalladas tanto el laboratorio como las ataaques principales objeto de estudio, se muestran las diferentes pruebas realizas en las versiones de Windows especificadas en el estado del arte.\\

En el Capítulo 6 se presentan y se discuten los resultados obtenidos.\\

En el Capítulo 7, para finalizar el proyecto, se lleva a cabo una reflexión sobre el esfuerzo realizado y sus diferentes líneas de trabajo futuro. \\


\chapter{Autenticación y autorización en Windows}
Este capítulo aborda los términos y definiciones a tener en cuenta y sirve como introducción a Active Directory. En primer lugar, se define la forma en la que los Sistemas Windows gestioana la autenticación y la auterización. Posteriormente, se definen los protocolos de seguridad para la verificación de la autenticación NT Lan Manager y Kerberos. Finalmente, se presenta Active Directory y la terminología necesaria relativa a este.\\

\section{Autenticación y Autorización}

Uno de los principales requisitos a la hora de entender como funcionan la mayoría de los ataques contra Sistemas Windows pasa por la gstión de la autenticación y autorización de los usuarios que inician sesión en el ordenador ya sea a nivel local o en red.\\

Por un lado, la \textbf{autenticación} consiste en la verificación de la identidad de un usuario, dicho con otras palabras, que el sistema de autenticación se asegure de que un usuario es quién dice ser. Por ejemplo, conociendo la contraseña del usuario que dice ser.\\

Por otro lado, la \textbf{autorización} consiste en establacer y delimitar los recursos a los que puede acceder, o no puede acceder ya que los tiene restringidos un usuario (o grupos de usuarios).\\


\subsection{Inicio de sesión interactivo (Interactive Logon)}

El proceso de autenticación a través de inicio de sesión interactivo del inglés {\it Interactive Logon}, a diferencia del inicio de sesión en red o {\it Network Logon}, es llevado a cabo por el proceso {\it WinLogon} que se encarga de recoger las credenciales introducidas por el usuario y su posterior validación. Un usuario que inicia sesión en un equipo ya sea localmente o un inicio de sesión en red introduce el usuario y la contraseña (denominado credenciales de usuario) y sirve para verificar la identidad del usuario. Por otro lado, cuando se inicia sesión a través de una Smart Card {\it (Smart Card Logon)} las credenciales están almacenadas en el chip de la tarjeta y estas son leídas por un dispositivo externo y el usuario introduce el {\it Personal Identification Number (PIN)}~\cite{Capitulo2:Logon}.\\

\subsubsection{Proceso WinLogon}

WinLogon es el proceso encargado de coordinar el inicio de sesión. Además, este proceso también se encarga de gestionar el {\it logout}, lanzar los procesos necesarios para la autenticación de un usuario, cambiar las contraseñas, bloquear y desbloquear un equipo y proporcionar la seguridad necesaria para que ningún otro proceso pueda acceder a información sensible cuando estos procedimientos se están llevando a cabo. 

Como se puede ver en la Figura \ref{WinLogon} el proceso de inicio iterativo consta de varias fases~\cite{Capitulo2:WinInternals}:

\begin{figure}[t!] %[ht!] para here [b] para bottom [t] para top
\begin{center}
\includegraphics[width=16cm]{WinLogon.png}
\end{center}
\caption{Proceso de inicio de sesión interactivo (WinLogon).}
\label{WinLogon}
\end{figure}

\begin{enumerate}
\item En primer lugar, el proceso de inicio de sesión comienza con una secuencia denominada {\it Secure Attention Sequence (SAS)}, esta secuencia es {\it CTRL + ALT + DEL} por defecto e inicia el proceso WinLogon.
\item Una vez iniciado el proceso WinLogon, este ejecuta el proceso {\it LogonUI} que proporciona la interfaz por defecto para introducir las credenciales y a su vez carga las bibliotecas de enlace dinámico, del inglés {\it Dynamic-Link Library (DLL)} que se encargan de recoger las credenciales y pasarlas al proceso denominado Servicio Subsistema de Autoridad de Seguridad Local del inglés {\it Local Security Authority Subsystem Service} (LSASS). Estas DLLs denominadas {\it Credential Providers} se encuentran en \footnote{\%SystemRoot\%\textbackslash{}System32\textbackslash{}authui.dll} o en \footnote{\%SystemRoot\%\textbackslash{}System32\textbackslash{}SmartcardCredentialProvider.dll} (Si se trata de un inicio de sesión con Smart Cards).
\item Al ejecutarse Winlogon, también se crea un número identificador de seguridad del inglés {\it Security Identifier} (SID), este número se pasa como argumento en la llamada {\it LsaLogonUser} y será incluido en el Token de Acceso {\it (Access Token)} si la autenticación se procesa correctamente. 
\item Una vez introducido usuario y contraseña, WinLogon llama al proceso LSASS a través de la función {\it LsaLookupAuthenticationPackage}. Esta función tiene como objetivo obtener los paquetes de autenticación disponibles en el sistema a través de la clave de registro \footnote{HKEY\_LOCAL\_MACHINE\textbackslash{}SYSTEM\textbackslash{}CurrentControlSet\textbackslash{}Control\textbackslash{}Lsa} como se puede observar en la Figura \ref{Registro-Auth}.

\begin{figure}[t!] %[ht!] para here [b] para bottom [t] para top
\begin{center}
\includegraphics[width=16cm]{Registro-Auth.jpg}
\end{center}
\caption{Clave de registro sobre los paquetes de autenticación.}
\label{Registro-Auth}
\end{figure}

\item Posteriormente, se envían las credendiales a través de la función {\it LsaLogonUser}. Si algún paquete de autenticación autentica el usuario el proceso continua, en cambio, si ningún paquete indica que se ha iniciado sesión correctamente el proceso acaba.

\item Una vez autenticado, el proceso LSASS comprobará en la base de datos de políticas locales si el usaurio autenticado tiene los permisos suficientes para realizar la acción que está solicitando. Si el inicio de sesión no coincide el proceso de autenticación acaba y LSASS elimina cualquier estructura de datos creada y lo notifica a WinLogon. Si el acceso está permitido, LSASS agrega los IDs de seguridad correspondientes, busca en la base de datos los permisos asociados a los usuarios del mismo grupo del SID del usuario y los añade al token de acceso ({\it Access Token}) y crea el Token que será enviado a Winlogon con un mensaje de inicio de sesión correcto. 

\end{enumerate}

Una vez definido el proceso de inicio interactivo a grandes rasgos, se va a pasar a detallar los componentes mencionados que forman parte de dicho proceso.

\subsubsection{Paquetes de Autenticación (Authentication Package)}

\section{NT Lan Manager}

\section{Kerberos}

\section{Active Directory}




\chapter{Paquetes de autenticación}
Esta sección detalla en profundidad los paquetes de autenticación utilizados en sistemas basados en Windows. Los paquetes de autenticación son {\it Dynamic-Link Libraries (DLLs)} lanzadas por el proceso LSA durante un inicio de sesión que se encargan de analizar y validar las credenciales introducidas por el usuario, crear una nueva {\it logon session} y pasar la información al proceso LSA para que este cree el {\it Access Token} correspondiente si la validación ha sido correcta. Windows permite la carga de multiples paquetes de autenticación lo que permite que LSA siporte multiples procesos de inicio de sesión diferentes. En este capítulo se van a detallar los paquetes de autenticación utilizados por defecto: MSV1\_0 y Kerberos. Además, se va a detallar la forma que tiene Windows de almacenar las contraseñas en el sistema.

\section{MSV1\_0}

MSV1\_0~\cite{Capitulo3:MSV10} es el paquete de autenticación proporcionado por Windows e implementa la familia de protoclos Lan Manager versión 1 y 2 (LM y NT) y Net Lan Manager versión 1 y 2 (NTLMv1 y NTLM v2)~\cite{Capitulo3:NTLM}. \\

Este paquete de autenticación soporta tanto inicio de sesión de forma local como inicio de sesión para cuentas y servicios en dominios. El paquete MSV1\_0 ejecuta una aquitectura cliente/servidor, es decir, el cliente es el que recibe las credenciales (username y el hash de la contraseña) y las valida frente al servidor. \\

Cuando se ejecuta localmente cliente y servidor están representados por la misma máquina que se encarga de recoger las credenciales proporcionadas por el usuario a través de los {\it Credential Providers} y compararlas con las credenciales introducidas por el usuario cuando creó la cuenta y que están almacenadas en la SAM, si ambas contraseñas son iguales el proceso de autenticación es correcto. \\

En inicio de sesión en dominio el cliente representa la máquina local y el servidor representa el Domain Controller donde está configurado Active Directory. El cliente recoge las credenciales y las pasa por el canal de comunicación seguro creado por el proceso {\it Winlogon.exe} y las comunica con la instancia de MSV1\_0 ejecutada en el Domain Controller. El cliente delega la comprobación de las credenciales al Domain Controller, esto se denomina {\it Pass-Through}. La instancia de MSV1\_0 del Domain Controller realiza la validación de las credenciales comprobando la información recibida con los datos almacenados el la base de datos del Domain Controller y devuelve la información a la instancia ejecutada en local. Si la validación ha sido correcta, el paquete MSV1\_0 local devuelve la información al proceso LSA local. \\

Windows ha implementado los protocolos de desafío/respuesta NTLMv1 y NTLMv2 para la intercambiar las credenciales introducidas por el usuario entre la máquina local y el Domain Controller en lugar de intermbiar las credenciales directamente. Antes de detallar ambos protocolos, se va a analizar la forma de almacenamiento de las contraseñas que luego serán utilizadas por ambos protocolos. 


\subsection{Windows hashes}

Los Sistemas Windows, en lugar de almacenar las contraseñas en texto plano, algo que sería un gran problema de seguridad utilizan los siguientes algoritmos de hash~\cite{Capitulo3:Hashes}:

\subsubsection{Hashes Lan Manager (LM)}

El algoritmo Lan Manager (LM) para realizar la función hash de las contraseñas almacenadas en Windows fue una de las primeras implementaciones que desarrolló Windows para mantener cifradas las contraseñas. Hoy en día está práctiacamente en desuso y desde 2017 se recomienda desactivar la opción de que se guarden las credenciales de esta forma~\cite{Capitulo3:LMDeprecated}. 

\textbf{Algoritmo}

El algoritmo de hash utilizado realiza el siguiente procedimiento: 

\begin{itemize}
\item Convertir todos los carácteres a letras mayúsculas.
\item Añadir un padding de carácteres nulos hasta que tenga una longitud de 14 carácteres. 
\item Dividir la contraseña en dos partes de 7 carácteres cada una. 
\item Crear dos DES keys para cada parte. 
\item Cifrar a través de DES las partes anteriores con el string "KGS!@\#\$\%".
\item Concatenar ambos strings.
\end{itemize}

\subsubsection{Hashes NT}

Los hashes NT, también conocidos como hashes NTLM, es la forma que utiliza atualmente Windows para alamcenar las contraseñas de los usuarios del sistema. Estos hashes están almacenados en la SAM si se trata de un equipo local o en el fichero NTDS del Active Directory si se trata de un equipo en dominio. A través de la obtención de este tipo de hashes se puede realizar un ataque de Pass-The-Hash (se detallara en los siguientes capítulos).

\textbf{Algoritmo}

Windows encodea la contraseña del usuario con UTF-16 Little Endian y posteiormente realiza un hash con el algoritmo MD4: 

\begin{itemize}
\item MD4(UTF-16-LE(password))
\end{itemize}

\subsection{Net-NT Lan Manager (Net-NTLM)}

El protocolo Net-NTLM es un protocolo {\it challenge/response} utilizado para la autenticación entre el cliente y el servidor~\cite{Capitulo3:NTLM2}. El objetivo principal de este protocolo es proporcionar la autenticación de un dispositivo sin la necesidad de intercambiar implícitamente la contraseña con el servidor. Además, este protocolo proporcionada integridad y confidencialidad ya que los mensajes intercambiados van cifrados. Windows ha proporcioado dos implementaciones de este protocolo~\cite{Capitulo3:NTLANManager}:

\subsubsection{Net NT Lan Manager Versión 1 (Net-NTLMv1)}

Es la versión más antigua de este protocolo y actualmente se encuentra en desuso ya que presenta limitaciones de gran importancia. Este protocolo utiliza ambos de los hashes explicados en la sección anterior (LM y NT). A continuación se va a detallar protocolo utilizado:

\begin{enumerate}
\item El cliente realiza una petición de autenticación al servidor.
\item El servidor responde con un challenge que corresponde a un número aleatorio de 8 bytes. 
\item El cliente realiza una operación criptográfica utilizando el challenge enviado por el servidor y un secreto que ambos conocen, en este caso se va a utilizar alguno de los windows hashes explicados anteriormente (o los dos). El cliente enviará al servidor el resultado de esta operación (24 bytes). 
\item El servidor comprueba si se ha realizado la operación correctamente ya que también dispone tanto el challenge como el secreto utilizado. Si el challenge coincide la autenticación se ha realizado correctamente. 
\end{enumerate}

\subsubsection{Net NT Lan Manager Versión 2 (Net-NTLMv2)}

Debido a las limitaciones que presentaba NTMLv1, Windows implementó una versión mejorada de este protocolo: NTLMv2 que está disponible desde el paquete Windows NT 4.0 SP4. De la misma manera, se va a explicar el protocolo {\it challenge/response} utilizado para esta versión:

\begin{enumerate}
\item El cliente realiza una petición de autenticación al servidor. 
\item El servidor responde con un challenge que corresponde a un número aleatorio de 8 bytes.
\subitem - Server Challenge (SC) = 8-byte challenge (Random).
\item El cliente genera también un numbero aleatorio de 8 bytes.
\subitem - Client Challenge (CC) = 8-byte challenge (Random).
\item El cliente calcula el secreto que va a utilizar a través de realizar el algorimo HMAC-MD5 del hash NT de la contraseña, el nombre de usuario y el dominio. 
\subitem v2-Hash = HMAC-MD5(NT-Hash, user name, domain name)
\item El cliente envía dos respuestas diferentes: 
\subitem - LMv1: Que corresponde con el hash HMAC-MD5 del v2-hash y los dos challenges (SC y CC): LMv2 = HMAC-MD5(v2-Hash, SC, CC)
\subitem - NTv2: Que corresponde con el hash HMAC-MD5 del v2-hash, el challenge del sevidor y un nuevo challenge del cliente que incluye un timestamp para evitar ataques de replay: CC* = (X, time, CC2, domain name) --- NTv2 = HMAC-MD5(v2-Hash, SC, CC*)
\item El servidor comprueba si se las operaciones se han realizado correctamente ya que también dispone tanto el challenge como el secreto utilizado. Si el challenge coincide la autenticación se ha realizado correctamente.
\end{enumerate}

\section{Kerberos}

El paquete de autenticación Keberos, que implementa la versión 5 del protocolo de Kerberos~\cite{Capitulo3:Kerberos}, es el paquete principal utilizado por los sistemas Windows para verificar la identidad de un equipo cuando se realiza un inicio de sesión en red. Las principales características de este protocolo son~\cite{Capitulo3:Kerberos2}: 
\begin{itemize}
\item Proporcionar autenticación a través del uso de tickets.
\item Evitar el intercambio o almacenamiento de credenciales. 
\item La utilización de un tercero de confianza ({\it trusted 3rd-party}).
\item La utilización de criptografía simétrica. 
\end{itemize}

\subsection{Aplicaciones de Kerberos}

Las ventajas de utilizar Kerberos como protocolo de autenticación son las siguientes~\cite{Capitulo3:Kerberos3}:

\begin{itemize}
\item \textbf{Autenticación delegada:} La autenticación con Kerberos permite a un servicio actuar impersonando al cliente local cuando se conecta a otros servicios.
\item \textbf{Single Sig-On:} El uso de Kerberos permite a los usuarios acceder a los recursos de un dominio sin introducir la contraseña cada vez que quieran acceder a un recurso diferente.
\item \textbf{Autenticación eficiente:} El servidor donde se está intentado loguear un equipo tiene la capacidad de autenticar a este examinando únicamente las credenciales presentadas por el cliente. Es decir, un cliente puede obtener las credenciales para un servidor en partciular una vez y reutilizarlas. 
\item \textbf{Autenticación mutua:} A diferencia de NTLM, Kerberos puede autenticar ambas partes, tanto el cliente como el servidor. 
\end{itemize}

\subsection{Elementos principales}

Antes de explicar el procedimiento utilizado por el paquete de autenticación Kerberos, es necesario detallar los diferentes elementos que van a formar parte del mismo~\cite{Capitulo3:Kerberos4}. 

\subsubsection{Ticket-Granting Ticket (TGT)}

{\it Ticket-Granting Ticket (TGS)} corresponde con un identificador cifrado con un tiempo de uso limitado expedido por el Key Distribution Center (KDC) cuando se ha completado la autenticación de un usuario y sirve para solicitar los {\it Ticket-Granting Server (TGS)} cuando se quiere utilizar dicho servicio. Este ticket está cifrado con la clave del KDC. El tiempo de validez por defecto de un ticket es de diez horas. 

\subsubsection{Ticket-Granting Server (TGS)}

{\it Ticket-Granting Server (TGS)}~\cite{Capitulo3:TGT} es el identificador que un usuario presenta a un servicio para poder acceder a sus recursos. Para solicitar este ticket el usuario presenta el {\it Ticket-Granting Ticket (TGS)} para verificar la validez de la autenticación y si tiene permisos de acceso a este recurso. Este ticket etá cifrado con la clave del servicio correspondiente.

\subsubsection{Key Distribution Center (KDC)} 

{\it Key Distribution Center (KDC)} es el servicio encargado de recibir las peticiones de autenticación, validar los datos contenidos en esta y si la autenticación es correcta proporcionar un {\it Ticket-Granting Ticket (TGT)}. Este proceso se ejecuta en el Domain Controller que administra el Active Directory. \\

Este elemento se puede dividir en dos instancias principales: El servidor de autenticación ({\it Authentication Server}) y el servidor que gestiona los TGTs ({\it Ticket Granting Server}). 

\begin{itemize}
\item \textbf{Authentication Server (AS):} Proporciona la autenticación de un usuario en la red y genera el ticket TGT. 
\item \textbf{Ticket Granting Server (TGS):} Cuando un usuario solicita el acceso a un servicio red, presenta el ticket TGT y este le proporciona un ticket TGS que sirve de autenticación frente al servicio de red destino. 
\end{itemize}

\subsubsection{Application Server (AP)}

{\it Application Server (AS)} corresponde con cualquier aplicación que soporte autenticación a través del protocolo Kerberos. Corresponde al servicio o recurso al que quiere acceder el cliente. 

\subsubsection{Claves}

En el protocolo de autenticación Kerberos hay tres claves fundamentalmente: 
\begin{itemize}
\item \textbf{Clave del KDC o krbtgt:} Clave derivada del hash NTLM de la cuenta {\it krbtgt}, sirve para cifrar las partes más importantes del protocolo como el TGT.
\item \textbf{Clave del cliente: } Clave derivada del hash NTLM del usuario o cliente. 
\item \textbf{Clave del servicio: } Esta clave depende del servicio y es la que se utiliza para cifrar los tickets TGS. 
\end{itemize}

También existen diferentes claves de sesión negociadas entre el KDC y el cliente y claves de sesión del servicio negociada entre el cliente y el AS. 

\subsubsection{Privilege Attribute Certificate (PAC)}

{\it Privilege Attribute Certificate (PAC)}~\cite{Capitulo3:PAC} es una estructura de datos que recoge la información codificada sobre los privilegios del usuario. Esta estrucura está cifrada con la clave del KDC. El cliente puede especificar que no se incluya el PAC en la petición del TGT. Un servicio puede comprobar con el KDC si el PAC está firmado correctamente. 

\subsection{Protocolo de autenticación}

En esta sección se va a explicar el protocolo de autenticación, para este caso de usuo el cliente solicita un TGT para autenticarse y posteriormente utiliza ese TGT para pedir un ticket de servicio para poder acceder a una aplicación. Se va a detallar el proceso utilizado y los paquetes intercambiados~\cite{Capitulo3:Kerberos5}. Cabe destaca que Kerberos utiliza  TCP y UDP para el intercambio de paquetes. El KDC utiliza los puertos TCP/88 y UDP/88.   

\begin{figure}[t!] %[ht!] para here [b] para bottom [t] para top
\begin{center}
\includegraphics[width=16cm]{Kerberos.png}
\end{center}
\caption{Protocolo de autenticación Kerberos.}
\label{Kerberos}
\end{figure}

\begin{enumerate}

\item En primer lugar, el cliente envía una petición de autenticación al servidor KDC a través del paquete \textbf{KRB\_AS\_REQ}. El objetivo de este paquete es iniciar la comunicación y transmitir las credenciales del usuario a autenticar. Para ello se transmite la siguiente información:

\textbf{- Timestamp:} Sirve para evitar ataques de replay. Está firmado con la clave NTLM del cliente.\\
\textbf{- Username:} Información sobre el nombre del usuario que se está autenticado.\\
\textbf{- Service Principal Name (SPN)~\cite{Capitulo3:SPN}:} Indicador único de la instancia del servicio asociado a la cuenta krbtgt. \\
\textbf{- Nonce:} Número aleatorio generado por el usuario. \\

\item El {\it Authentication Server (AS)} recibe el paquete anterior y procede a la autenticación. Para ello busca el nombre de usuario en la base de datos del KDC y utiliza el hash de la contraseña almacenada para descifrar el timestamp, si no se produce ningún error al descifrar y el timestamp coincide con la hora actual (con un desfase máximo de 5 minutos) la autenticación se completa correctamente. \\

Una vez autenticado el usuario, el AS prepara el paquete a enviar denominado \textbf{KRB\_AS\_REP}. Este paquete contiene la siguiente información:

\textbf{- Username:} Información sobre el nombre del usuario que se está autenticado.\\
\textbf{- Datos cifrados:} Información cifrada con la clave del usuario que incluye: Nombre del usuario, clave de sesión, fecha de expiración de la sesión, SPN y el nonce enviado por el cliente previamente. \\
\textbf{- Ticket TGT:} Ticket cifrado con la clave del KDC. El ticket incluye: Nombre del usuario, clave de sesión, fecha de expiración del ticker TGT y PAC. \\

\item Una vez autenticado y en disposición del TGT. Para poder utilizar un servicio es necesario obtener un TGS. Para ello, el usuario envía un paquete \textbf{KRB\_TGS\_REQ} con la siguiente información:

\textbf{- SPN:} Indicador único de la instancia del servicio asocido a la cuenta krbtgt. \\
\textbf{- Nonce:} Número aleatorio generado por el usuario. \\
\textbf{- Ticket TGT}
\textbf{- Datos cifrados:} Datos cifrados con la clave del usuario que incluye: Nombre de usuario y timestamp. \\

\item {\it Ticket Granting Server} examina la petición, si esta es correcta envía el paquete \textbf{KRB\_TGS\_REP} con el TGS y la siguiente información:

\textbf{- Username}. \\
\textbf{- Ticket TGS:} Ticket cifrado con la clave del servicio. El ticket incluye: Clave de sesión del servicio, nombre de usuario, fecha de expiración del ticket TGS y PAC. \\
\textbf{- Datos cifrados:} Información cifrada con la clave de sesión que incluye: Clave de sesión del servicio, fecha de expiración del ticket TGS y Nonce enviado previamente.

\item Una vez obtenido el ticket TGS, el usuario podrá presentarlo al servicio correspondiente, para ello debe enviar a dicho servicio el paquete \textbf{KRB\_AP\_REQ} con la siguiente información: 

\textbf{- Ticket TGS}.\\
\textbf{- Datos cifrados:} Información cifrada con la clave de sesión del servicio que incluye: Nombre de usuario y timestamp. \\

\item Por último, el servidor contesta con el paquete \textbf{KRB\_AP\_REP}. Este paquete es opcional y sólo se envía si es necesaaria la autenticación mutua entre el cliente y el servicio. 

\end{enumerate}



\chapter{Active Directory}
Directotio Activo del inglés {\it Active Directory (AD)}, corresponde a la 

Active Directory, como ya se ha comentado en la introducción a este trabajo, es el entorno más utilizado por empresas y organizaciones a nivel mundial para la gestión y administración centralizada de redes y recursos. Para ello, Microsoft, desde Microsoft Windows Server 2000, propone un servicio de directorio utilizado para gestionar la in\-for\-ma\-ción sobre todos los recursos en red que dispone una empresa y permite a los usuarios y grupos de usuarios aceeder de forma jerárquica a dichos recursos. La in\-for\-ma\-ción admisitrada por Active Directory se puede agrupar en tres grupos princiaples: recursos (impresoras), servicios (aplicaciones web, aplicaciones de correo electróncio, bases de datos, etc.) y usuarios (cuentas, credenciales, grupos, etc.). Con esta in\-for\-ma\-ción, es posible crear y adminsitrar dominios, usuarios y todos los objetos englobados dentro de la misma red. En este capítulo se va a introducir los términos generales y conceptos clave y la creación de un laboratorio local que permita la ejecución de los principales ataques sobre Active Directory.






\section{Términos y conceptos clave}


\subsubsection{Active Directory Domain Services (ADDS)}
\subsubsection{Dominio (Domain)}

Un dominio, del inglés {\it Domain}~\cite{Capitulo4:Domain} es definido como un ``contenedor lógico'', es decir, es una estructura lógica que contiene los siguientes componentes:

\begin{itemize}
\item Una estructura jerárquica para usauarios y grupos en función de los privilegios de los mismos.
\item Servicios (vistos anteriormente) que proveen capacidades de autenticación y autorización.
\item Distintas políticas de seguridad que se aplican a usuarios y objetos.
\item Un registro DNS que identifica inequívocamente el dominio, como puede ser empresa.com, ad.empresa.com. Este nombre será requisito para iniciar sesión en una cuenta de dominio utilizándose como parte del nombre del usuario. 
\end{itemize}

Estos componentes y objetos están almacenados en la base de datos de Active Directory. Se puede considerar un dominio como un límite administrativo de estos objetos. Un dominio puede abarcar diferentes ubicaciones tanto físicas como en red y estar compuesto por una multitud de objetos.

\subsubsection{Árbol (Domain Tree)}

En relación con el término anterior, un árbol del inglés {Domain Tree}, son colecciones de dominios que se agrupan como una estructura jerárquica. Un arbol se le puede considerar como una serie de dominios conectados jerárquicamente a través de usar el mismo espacio de nombres DNS. Un ejemplo sería, si al dominio anterior: empresa.com le añadimos un ``hijo'' denominado recursoshumanos.empresa.com se crea un árbol de dominios compuesto por un dominio padre o root (empresa.com) y un hijo o child (recursoshumanos.empresa.com). Estos dominios forman parte del mismo árbol y se crean automáticamente relaciones de confianza entre ellos. En un Active Directory pueden coexisstir multitud de árboles de dominio diferentes. 

\subsubsection{Bosque (Forest)}

Un bosque, del inglés {\it Forest}, a grandes rasgos es una colección de árboles de dominio que comparten el mismo {\it schema}, misma estructura lógica, {\it global catalog} y configuración. Alguno de estos términos será introducido a continuación. Todos los dominios pertenecientes a un mismo forest, establecen una relación de confianza transitiva. Cabe destacar, que cuando se crea una instacia de Active Directory por primera vez y se crea un dominio también se está creando implícitamente un forest. 

\subsubsection{Schema}

Un {\it schema} en Active Directory se define como a {\it forest-wide template}, es decir, una plantilla aplicable al dominio que define los objetos y propiedades alojados en el Active Directory. Este esquema debe estar bien configurado para evitar comprometer la seguridad de todos los dominios del forest. Para su administración existe un grupo especial denominado {\it Schema Admins} que puede editar y configurar dicho schema. 

\subsubsection{Fully Qualified Domain Name (FQDM)}

Fully Qualified Domain Name (FQDN) es la dirección completa que identifica un host o recurso, este está compuesto por la unión del nombre del host {\it hostname} y el dominio. En el ejemplo anterior, un equipo denominado Cliente01 su FQDN sería Cliente01.empresa.com. 

\subsubsection{Domain Controller (DC)}

Un controlador de dominio, del inglés {\it Domain Controller (DC)}, es la parte fundamental de Active Directory, corresponde a servidores de Windows que contienen la base de datos Active Directory y por lo tanto almacenan toda la información correcpondiente a dominios, domain trees, forests, usuarios, servicieos, etc. 

\subsubsection{Objetos (Objects)}

Todos los elementos alamacenados en una base de datos Active Directory se almacenan en forma de objetos, cada objeto tiene un tipo diferente que le diferencia de otros objetos. Cada objeto almacenado tiene un SID diferente que se utiliza para admitir o denegar el acceso del objeto a un recurso del dominio. Los objetos creados por defecto en cualqueir dominio se pueden agrupar de la siguiente forma: 

\begin{itemize}
\item Unidades organizativas, del inglés {\it Organizational Unit (OU)}. 
\item Usuarios.
\item Ordenadores. 
\item Grupos de usuarios.
\item Contactos.
\item Carpetas compartidas. 
\item Impresoras compartidas.
\end{itemize}

\subsubsection{Organizational Unit (OU)}

Unidades organizativas, del inglés {\it Organizational Unit (OU)} son contenedores de diferentes objetos del mismo dominio como puede ser otros contenedores, cuentas de usuario, grupos, etc. Un administrador del dominio puede crear unidades organizativas y aplicarle diferentes directivas de grupo que se aplicarán a todos los objetos de esta unidad, esto permite una administración más eficiente del Active Directory. 

\subsubsection{Service Principal Name (SPN)}

Un {\it Service Principal Name (SPN)}~\cite{Capitulo4:SPN} es un identificador único asociado a una instacia de un servicio. Los SPN son utilizados por el protocolo de autenticación Kerberos para asociar una instacia de un servicio en concreto con una cuenta de inicio de sesión.  

\subsubsection{Global Catalog}


\section{Laboratorio de Active Directory}

En la siguiente sección se van a establecer las directrices para la creación de un laboratorio local que permita la realización de los ataques que se describirán y expemirentación en los capítulos previos a este. La organización de este capítulo es la siguiente: En primer lugar se detallan los requisitos o prerrequisitos necesarios para poder realizar las siguientes acciones como puede ser el software de virtualización, las imágenes del sistema operativo, etc. Posteriormenente, se ha definido y configurado la topología elegida y por último la instalación y administración de Active Directory. 

\subsection{Requisitos}

Previamente a la creación de la topología de red y a la instalción de  Active Directory que permita realizar las pruebas es necesario disponer de las siguientes características.

\subsubsection{Software de virtualización}

La virtualización consiste en la creación de entornos simulados o recursos desde un único sistema operativo denominado {\it host}, por consecuencia, un software de virtualización es aquel que te permite realizar las acciones descritas anteriormente que puede ser la creación de sistemas operativos, creación de topologías de red, administración de recursos etc. Aunque hay gran variadad de sotfware de virtualización, para la realización de este proyecto se ha utilizado {\it Oracle VM VirtualBox}~\cite{Capitulo4:VirtualBox}.\\

{\it VirtualBox} es un software de virtualización {\it Open Source} con licencia GPLv2 desarrollado por Oracle Corporation que permite la creación de entornos x86 and AMD64/Intel64. Para este proyecto se ha utilizado la última versión (VirtualBox 6.0.12) que se puede desacargar en \footnote{https://www.virtualbox.org/}. Con este Software se van a crear las máquinas virtuales y las redes internas necesarias para la creación del laboratorio. 

\subsubsection{Máquinas Virtuales}

Para este proyecto se van a utilizar cuatro máquinas virtuales. Para aquellas que se necesite una licencia de software privativo se va a utilizar la versión de prueba que proporciona Microsoft con el objetivo de que cualquier pueda replicar dicho laboratorio sin necesidad de lincencias adicionales. Las máquinas son las siguientes:

\begin{itemize}
\item \textbf{DC01}: El la máquina virtual principal y es la encargada de administrar el Active Directory. Como se ha comentado en el estado del arte se va a utilizar Windows Server 2019 es su última versión. La imagen del sistema operativo se ha descargado de \footnote{https://www.microsoft.com/en-us/evalcenter/evaluate-windows-server-2019}.

\item \textbf{Cliente01}: Por otro lado, esta máquina representa a la de un usuario legítimo o cliente de una empresa que está unido al dominio y conectado por la red interna. Para esta máquina virtual se ha utilizado Windows 10 Enterprise que se puede descargar en \footnote{https://www.microsoft.com/en-us/evalcenter/evaluate-windows-server-2019}. 

\item \textbf{Gateway}: Esta máquina virtual se va a utilizar como puerta de enlace entre la red interna y la red externa lo que simula ser internet. Para su implementación, se ha utilizado Debian 10 (Buster) sin escritorio para ahorrar recursos locales. La imagen de este sistema operativo se puede descargar en \footnote{https://cdimage.debian.org/debian-cd/current/amd64/iso-cd/debian-10.1.0-amd64-netinst.iso}.

\item \textbf{Atacante01}: Por último, para la simulación de un atacante externo o profesional de la seguirdad ofensiva realizando labores de {\it Red Team}, se ha utilizado la distribución Kali Linux 2019.3, este distribución ofrece gran variedad de herramientas destinadas a la auditoría informática que serán de utilidad a la hora de realizar los ataques propuestos. Para descargar esta distribución se puede a través de \footnote{https://cdimage.kali.org/kali-2019.3/kali-linux-2019.3-amd64.iso}.

\end{itemize}


\subsubsection{Instalación y actualización}

La instalación y actualización de las máquinas, no se ha considerado como alcance de este proyecto, por lo tantoa partir de este punto se da por hecho de que el usuario ha instalado y actualizado las máquinas y cuenta con la última versión de las mismas. 

\subsection{Configuración previa}

Antes de configurar el active directory, se ha creado una topología en red simula a un entorno coorporativo fictio. Aunque este laboratorio únicamente disponga de una máquina unida al dominio, en entornos reales son multitud los equipos lo que posibilita un gran abanico de posibles entradas a la red de la empresa u organización. A contanuación se va a explicar la topología elegida y las configuraciones necesarias.. 

\subsubsection{Topología de red}

Como ya se ha adalentado, la máquina consta de 4 máquinas: 2 Windows (DC01 y Cliente01) y 2 Linux (Atacante01 y Gateway). La distribución de la topología en red se puede observar en la Figura \ref{Topología}. En la imagen se puede apreciar la existencia de dos redes: ADNET(192.168.0.0/24) formada por los dos dos Sistemas Windows que forman parte del dominio de la empresa fictia y EXTNET(10.10.10.0) que emula en una red interna lo que sería estar expuesto a internet en un entorno real. Ambas redes están enlazadas por el Gateway. 

\begin{figure}[t!] %[ht!] para here [b] para bottom [t] para top
\begin{center}
\includegraphics[width=16cm]{Topologia.png}
\end{center}
\caption{Topología del laboratorio local.}
\label{Topología}
\end{figure}

\subsubsection{Configuración de red}

Se va a realizar la configuración necesaria para cada red. 

\begin{itemize}
\item \textbf{DC01}
\begin{enumerate}
\item Antes de arrancar la máquina virtual, es necesario ir a Configuración/Red y añadir el Adaptador1. Esto va a simular la tarjeta de red del DC01. Esta tarjeta de red la vamos a conectar a la red ADNET como se puede ver en la Figura \ref{DC01-Red1}.

\begin{figure}[H] %[ht!] para here [b] para bottom [t] para top
\begin{center}
\includegraphics[width=16cm]{DC01/Red1.png}
\end{center}
\caption{Configuración de red DC01 - Tarjeta de red.}
\label{DC01-Red1}
\end{figure}

\item Una vez iniciada la máquina es necesario dirigirse a {\it Control Panel - Network and Internet - Network Connections} (Figura \ref{DC01-Red2}) y aparecerá la tarjeta de red añadida en el paso anterior. Para editar las direcciones es necesario entrar a las propiedades.
\begin{figure}[H] %[ht!] para here [b] para bottom [t] para top
\begin{center}
\includegraphics[width=16cm]{DC01/Red2.png}
\end{center}
\caption{Configuración de red DC01 - Ajustes de Ethernet.}
\label{DC01-Red2}
\end{figure}

\item Por último, como se puede observar en la Figura \ref{DC01-Red3}, se elige {\it Internet Version Protocol 4(TCP/IPv4) } y después las propiedades de este. Por último, se configura la dirección IP (192.168.0.3), la puerta de enlace correspondiente al Gateway (192.168.0.2) y el DNS. En la mayoría de entornos corporativos es el propio Active Directory el que hace la función de servidor DNS, por lo tanto se escribe la dirección de {\it loopback}: 127.0.0.1.
\begin{figure}[H] %[ht!] para here [b] para bottom [t] para top
\begin{center}
\includegraphics[width=16cm]{DC01/Red3.png}
\end{center}
\caption{Configuración de red DC01 - IPv4.}
\label{DC01-Red3}
\end{figure}
\end{enumerate}

\item \textbf{Cliente01}:
\begin{enumerate}
\item Para configurar la red en el Cliente01, al ser una Máquina Windows en la misma red que el DC01, es necesario repetir los mismos pasos que en la configuración anterior.
\item En el último paso, las direcciones IP son las siguientes: IP(192.168.04) y Gateway(192.168.0.2). En este caso la dirección DNS corresponde a la dirección IP del DC01: 192.168.0.3. La configuración resultante se puede ver en la Figura \ref{Cliente01-Red1}.

\begin{figure}[H] %[ht!] para here [b] para bottom [t] para top
\begin{center}
\includegraphics[width=10cm]{Cliente01/Red1.png}
\end{center}
\caption{Configuración de red Cliente01 - IPv4.}
\label{Cliente01-Red1}
\end{figure}
\end{enumerate}

\item \textbf{Gateway}:

\begin{enumerate}
\item Para esta máquina, es necesario habilitar dos interfaces, una que corresponde a la ADNET o red interna y otra que corresponde con la EXTNET o red externa (Figura \ref{Gateway-Red1}).
\begin{figure}[H] %[ht!] para here [b] para bottom [t] para top
\begin{center}
\includegraphics[width=10cm]{Gateway/Red1.png}
\end{center}
\caption{Configuración de red Gateway - Tarjetas de red.}
\label{Gateway-Red1}
\end{figure}

\item Iniciamos la máquina y comprobamos que se han creado ambas interfaces {\it enp0s3} para la ADNET y {\it enp0s8} para la EXTNET. 

\item A continuación introducimos los siguientes comandos, estos comandos levantan ambas interfaces y asignan las direcciones IP correspondientes. 
\begin{listing}[style=consola, numbers=none]
# ip link set enp0s3 up
# ip a add 192.168.0.2/24 dev enp0s3
# ip link set enp0s8 up
# ip a add 10.10.10.2/24 dev enp0s8
\end{listing}

\item Por último, permitimos que el gateway reenvíe los paquetes que le llegan, para eso se utiliza el siguiente comando: 
\begin{listing}[style=consola, numbers=none]
# echo 1 > /proc/sys/net/ipv4/ip_forward
\end{listing}

\item La configuración final de la máquina se puede observar en la Figura \ref{Gateway-Red2}.
\begin{figure}[H] %[ht!] para here [b] para bottom [t] para top
\begin{center}
\includegraphics[width=15cm]{Gateway/Red2.png}
\end{center}
\caption{Configuración de red Gateway}
\label{Gateway-Red2}
\end{figure}
\end{enumerate}

\item \textbf{Atacante01}:

\begin{enumerate}

\item Esta máquina está en la red externa, por lo tanto añadimos un adaptador de red unida a la red externa (Figura \ref{Atacante01-Red1}). 
\begin{figure}[H] %[ht!] para here [b] para bottom [t] para top
\begin{center}
\includegraphics[width=10cm]{Atacante01/Red1.png}
\end{center}
\caption{Configuración de red Atacante01 - Tarjeta de red}
\label{Atacante01-Red1}
\end{figure}

\item De la misma manera que en el gateway configuramos la interfaz de red, que en este caso es {\it eth0} con los siguientes comandos. 
\begin{listing}[style=consola, numbers=none]
# ip link set eth0 up
# ip a add 10.10.10.3 dev eth0 
\end{listing}

\item Para poder alcazar la red interna, es necesario definir a la dirección 10.10.10.2 como fateway, para que cuando la máquina no encuentre una dirección IP la redirija por ese Gateway. Para ello, se utiliza el siguiente comando: 

\begin{listing}[style=consola, numbers=none]
# ip route default via 10.10.10.2
\end{listing}

\end{enumerate}
\end{itemize}

\subsubsection{Comprobación de la conectividad}

\begin{itemize}

\item \textbf{Cliente01 - DC01}: Para comprobar la conectividad entre Cliente01 y DC10, podemos realizar un ping desde Cliente01 a la dirección IP de DC01 (\ref{Cliente01-Red2}).
\begin{figure}[H] %[ht!] para here [b] para bottom [t] para top
\begin{center}
\includegraphics[width=10cm]{Cliente01/Red2.png}
\end{center}
\caption{Conexión entre Cliente01 y DC01}
\label{Cliente01-Red2}
\end{figure}

\item \textbf{Atacante01 - DC01}: Para comprobar la conectividad dentre Atacante01 y DC10, en vez de realizar un ping intentamos conectarnos a través de Samba (\ref{Atacante01-Red2})
\begin{figure}[H] %[ht!] para here [b] para bottom [t] para top
\begin{center}
\includegraphics[width=10cm]{Atacante01/Red2.png}
\end{center}
\caption{Conexión entre Atacante01 y DC01}
\label{Atacante01-Red2}
\end{figure}

\end{itemize}

\subsubsection{Cambio de nombre del sistema}

Una buena práctica es cambiar el nombre a los Sistemas Windows, esto nos facilitará su identificación y su futura administración. 

\begin{itemize}
\item Para cambiar el nombre a DC01, se puede realizar desde el propio panel de administración del servidor, a través de la opción {\it Local Name Server - Computer Name - Change} como se puede ver en la Figura \ref{DC01-Name1}
\begin{figure}[H] %[ht!] para here [b] para bottom [t] para top
\begin{center}
\includegraphics[width=15cm]{DC01/Name1.png}
\end{center}
\caption{Cambio de nombre DC01}
\label{DC01-Name1}
\end{figure}

\item Para cambiar el nombre a Cliente01, se realiza a través de {\it Control Panel - System and Security - System} posteriormente se elige la opción {\it Change Settings - Change} y se introduce el nombre Cliente01 (\ref{Cliente01-Name1}). Ambos cambios nos van a pedir un reinicio del equipo.  
\begin{figure}[H] %[ht!] para here [b] para bottom [t] para top
\begin{center}
\includegraphics[width=15cm]{Cliente01/Name1.png}
\end{center}
\caption{Cambio de nombre Cliente01}
\label{Cliente01-Name1}
\end{figure}

\end{itemize}

\subsection{Creación y configuración del Active Directory}

En esta sección se va a detallar la instalación y configuración de Active Directory en Windows Server 2019 en la máquina virtual DC01.

\subsubsection{Intalación y configuración}

\begin{enumerate}

\item La instalación de un AD DS se puede realizar desde el Dashboard integrado en Windows Server 2019. Para ello, se selecciona {\it Add roles and features} (Figura \ref{DC01-AD1}).
\begin{figure}[H] %[ht!] para here [b] para bottom [t] para top
\begin{center}
\includegraphics[width=15cm]{DC01/AD1.png}
\end{center}
\caption{Instalación de AD DS - Add roles and features.}
\label{DC01-AD1}
\end{figure}


\item Para el tipo de instalación se va a elegir {\it Roled-based or feature-based installation} (Figura \ref{DC01-AD2}).
\begin{figure}[H] %[ht!] para here [b] para bottom [t] para top
\begin{center}
\includegraphics[width=15cm]{DC01/AD2.png}
\end{center}
\caption{Instalación de AD DS - Roled-based installation.}
\label{DC01-AD2}
\end{figure}


\item Después elegimos el DC01 (O el nombre que se haya elegido al cambiar el nombre en la sección anterior) como se puede ver en la Figura \ref{DC01-AD3}.
\begin{figure}[H] %[ht!] para here [b] para bottom [t] para top
\begin{center}
\includegraphics[width=15cm]{DC01/AD3.png}
\end{center}
\caption{Instalación de AD DS - DC01.}
\label{DC01-AD3}
\end{figure}


\item En la sección {\it Server Roles} se elige Active Directory Domain Services (Figura \ref{DC01-AD4}), al seleccionar esta opción se desplegará un menú donde debemos especifiar las caracterísitcas, se seleciona en {\it Add Features}.
\begin{figure}[H] %[ht!] para here [b] para bottom [t] para top
\begin{center}
\includegraphics[width=15cm]{DC01/AD4.png}
\end{center}
\caption{Instalación de AD DS - Active Directory Domain Services.}
\label{DC01-AD4}
\end{figure}


\item Después seguimos la instalción hasta la opción {\it Confirmation} e instalamos AD DS (Figura \ref{DC01-AD5}).
\begin{figure}[H] %[ht!] para here [b] para bottom [t] para top
\begin{center}
\includegraphics[width=15cm]{DC01/AD5.png}
\end{center}
\caption{Instalación de AD DS - Instalación.}
\label{DC01-AD5}
\end{figure}


\item Cuando se termina la instlación se debe ``promocionar'' el servidor DC01 como Domain Controller, para ello seleccionamos la opción que se puede ver en la Figura \ref{DC01-AD6}.
\begin{figure}[H] %[ht!] para here [b] para bottom [t] para top
\begin{center}
\includegraphics[width=15cm]{DC01/AD6.png}
\end{center}
\caption{Instalación de AD DS - Promote to Domain Controller.}
\label{DC01-AD6}
\end{figure}


\item Posteriormente, al no disponer de ningún forest previo, es necesario crear uno con el nombre {\it laboratory.com} (Figura \ref{DC01-AD7}). 
\begin{figure}[H] %[ht!] para here [b] para bottom [t] para top
\begin{center}
\includegraphics[width=15cm]{DC01/AD7.png}
\end{center}
\caption{Instalación de AD DS - Creación del forest laboratory.com.}
\label{DC01-AD7}
\end{figure}


\item En la pestaña {\it Domain Controller Options} elegimos Windows Server 2016 al ser la versión más actualizada posible, después si no se dispone de un DNS externo, se elige que el Domain Controllertenga la capacidad de DNS y Global Catalog. Además, se elige la contraseña para el Directory Services Restore Mode (DSRM) (Figura \ref{DC01-AD8}).
\begin{figure}[H] %[ht!] para here [b] para bottom [t] para top
\begin{center}
\includegraphics[width=15cm]{DC01/AD8.png}
\end{center}
\caption{Instalación de AD DS - Domain Controller options.}
\label{DC01-AD8}
\end{figure}


\item En la pestaña de {\it Paths} podemos ver las rutas de los principales elementos del DC como puede ser la base de datos NTDS y la carpeta SYSVOL (Figura \ref{DC01-AD9}).
\begin{figure}[H] %[ht!] para here [b] para bottom [t] para top
\begin{center}
\includegraphics[width=15cm]{DC01/AD9.png}
\end{center}
\caption{Instalación de AD DS - Rutas NTDS y SYSBOL.}
\label{DC01-AD9}
\end{figure}

\item Por último, instalamos las opciones definidas anteriormente Figura \ref{DC01-AD10}. Después de esta opción es necesario reiniciar el servidor. 
\begin{figure}[H] %[ht!] para here [b] para bottom [t] para top
\begin{center}
\includegraphics[width=15cm]{DC01/AD10.png}
\end{center}
\caption{Instalación de AD DS - Instalación.}
\label{DC01-AD10}
\end{figure}

\end{enumerate}

\subsubsection{Enlazar cliente al dominio}

Para añadir Cliente01 al dominio {\it laboratory.com} es necesario realizar los siguientes pasos: 

\begin{enumerate}
\item Del mismo modo que para cambiar el nombre al equipo, es necesario ir a {\it Control Panel - System and Security - System}, después realizar click en {\it Change settings} (Figura \ref{Cliente01-AD1}). 
\begin{figure}[H] %[ht!] para here [b] para bottom [t] para top
\begin{center}
\includegraphics[width=15cm]{Cliente01/AD1.png}
\end{center}
\caption{Enlazar cliente al dominio - Settings.}
\label{Cliente01-AD1}
\end{figure}

\item Después se selecciona {\it change} y en la opción de {\it Member of - Domain} se elige el dominio {\it laboratory.com} (Figura \ref{Cliente01-AD2}).
\begin{figure}[H] %[ht!] para here [b] para bottom [t] para top
\begin{center}
\includegraphics[width=15cm]{Cliente01/AD2.png}
\end{center}
\caption{Enlazar cliente al dominio - Dominio.}
\label{Cliente01-AD2}
\end{figure}


\item Al confirmar este cambio se requiere las credenciales del Domain Admin (Figura \ref{Cliente01-AD3}) y reiniciar el sistema.
\begin{figure}[H] %[ht!] para here [b] para bottom [t] para top
\begin{center}
\includegraphics[width=15cm]{Cliente01/AD3.png}
\end{center}
\caption{Enlazar cliente al dominio - Log on.}
\label{Cliente01-AD3}
\end{figure}

\item Finalmente, se puede confirmar que la operación se ha realizado correctamente desde el DC01 desde la opción {\it Tools - Active Directory Users and Computers - Laboratory.com - Computers} (Figura \ref{DC01-AD12}) del Dashboard como se puede ver en la Figura \ref{DC01-AD11}. 
\begin{figure}[H] %[ht!] para here [b] para bottom [t] para top
\begin{center}
\includegraphics[width=15cm]{DC01/AD12.png}
\end{center}
\caption{Enlazar cliente al dominio - Users and Computers.}
\label{DC01-AD12}
\end{figure}

\begin{figure}[H] %[ht!] para here [b] para bottom [t] para top
\begin{center}
\includegraphics[width=15cm]{DC01/AD11.png}
\end{center}
\caption{Enlazar cliente al dominio - Dashboard.}
\label{DC01-AD11}
\end{figure}


\end{enumerate}

\subsubsection{Creación de usuarios}

Para la fase de experimentación, además, se van a crear tres usuarios con distintos privilegios de administración en el dominio. Para ello, desde DC01 se elige la opción {\it Tools - Active Directory Users and Computers - Laboratory.com - Computers} (Figura \ref{DC01-AD12}) y después se selecciona {\it Users - New - User} como el la Figura \ref{DC01-AD13}. 

\begin{figure}[H] %[ht!] para here [b] para bottom [t] para top
\begin{center}
\includegraphics[width=15cm]{DC01/AD13.png}
\end{center}
\caption{Crear nuevo usuario.}
\label{DC01-AD13}
\end{figure}

\begin{itemize}

\item Usuario de dominio (Figura \ref{DC01-User1}). 
\begin{figure}[H] %[ht!] para here [b] para bottom [t] para top
\begin{center}
\includegraphics[width=10cm]{DC01/User1.png}
\end{center}
\caption{Usario de dominio.}
\label{DC01-User1}
\end{figure}


\item Usuario de dominio y administrador del dominio (Figura \ref{DC01-User1}). 
\begin{figure}[H] %[ht!] para here [b] para bottom [t] para top
\begin{center}
\includegraphics[width=10cm]{DC01/User2.png}
\end{center}
\caption{Usuario de dominio y administrador del dominio.}
\label{DC01-User2}
\end{figure}

Para añadirlo al grupo de administradores de dominio, seleccionamos la opción {\it Add to a group...} (Figura \ref{DC01-User3}).

\begin{figure}[H] %[ht!] para here [b] para bottom [t] para top
\begin{center}
\includegraphics[width=10cm]{DC01/User3.png}
\end{center}
\caption{Añadir el usuario a un grupo.}
\label{DC01-User3}
\end{figure}

Y se añade el grupo {\it Domain Admins} (Figura \ref{DC01-User4}).

\begin{figure}[H] %[ht!] para here [b] para bottom [t] para top
\begin{center}
\includegraphics[width=10cm]{DC01/User4.png}
\end{center}
\caption{Grupo Domain Admins.}
\label{DC01-User4}
\end{figure}

\item Usuario de dominio y administrador local en Cliente01 (Figura \ref{DC01-User5}).
\begin{figure}[H] %[ht!] para here [b] para bottom [t] para top
\begin{center}
\includegraphics[width=10cm]{DC01/User5.png}
\end{center}
\caption{Usuario de dominio y administrador local.}
\label{DC01-User5}
\end{figure}

Desde una consola con privilegios de administrador local en el Cliente01 ejecutamos el siguiente comando que añade al grupo de administradores el usuario creado previamente. 

\begin{listing}[style=consola, numbers=none]
# net localgroup administrators laboratory\mariarperez /add
\end{listing}

Una vez añadido, es posible loguearse con dicha información (Figura \ref{DC01-User6}).

\begin{figure}[H] %[ht!] para here [b] para bottom [t] para top
\begin{center}
\includegraphics[width=10cm]{DC01/User6.png}
\end{center}
\caption{Inicio de sesión con la cuenta de usuario creada..}
\label{DC01-User6}
\end{figure}


\end{itemize}





%----------
%	BIBLIOGRAFÍA
%----------	

%\nocite{*} % Si quieres que aparezcan en la bibliografía todos los documentos que la componen (también los que no estén citados en el texto) descomenta está lína

\clearpage
\addcontentsline{toc}{chapter}{Bibliografía}
%\setquotestyle[english]{british} % Cambiamos el tipo de cita porque en el estilo IEEE se usan las comillas inglesas.
%\printbibliography

\bibliography{Bibliografia/Bibliografia}{}
\bibliographystyle{ieeetr}


%----------
%	ANEXOS
%----------	

% Si tu trabajo incluye anexos, puedes descomentar las siguientes líneas
\chapter*{Instalación Windows Server 2019 (DC01)}
%\pagenumbering{gobble} % Las páginas de los anexos no se numeran
%\addcontentsline{toc}{chapter}{Anexo 1: Instalación Windows Server 2019 (DC01)}
%\textbf{Creación de Máquina Virtual}
\begin{enumerate}
\item Test.
\begin{figure}[H] %[H] para here [b] para bottom [t] para top
\begin{center}
\includegraphics[width=10cm]{DC01/MV1.png}
\end{center}
\end{figure}

\item Test
\begin{figure}[H] %[H] para here [b] para bottom [t] para top
\begin{center}
\includegraphics[width=10cm]{DC01/MV2.png}
\end{center}
\end{figure}

\item Test
\begin{figure}[H] %[H] para here [b] para bottom [t] para top
\begin{center}
\includegraphics[width=10cm]{DC01/MV3.png}
\end{center}
\end{figure}

\item Test
\begin{figure}[H] %[H] para here [b] para bottom [t] para top
\begin{center}
\includegraphics[width=10cm]{DC01/MV4.png}
\end{center}
\end{figure}

\item Test
\begin{figure}[H] %[H] para here [b] para bottom [t] para top
\begin{center}
\includegraphics[width=10cm]{DC01/MV5.png}
\end{center}
\end{figure}

\item Test
\begin{figure}[H] %[H] para here [b] para bottom [t] para top
\begin{center}
\includegraphics[width=10cm]{DC01/MV6.png}
\end{center}
\end{figure}

\item Test
\begin{figure}[H] %[H] para here [b] para bottom [t] para top
\begin{center}
\includegraphics[width=10cm]{DC01/MV7.png}
\end{center}
\end{figure}

\end{enumerate}



\textbf{Instalación}

\begin{enumerate}
\item Una vez creada la máquina virtual se ejecuta la instacia y se procede a la instalación.
\item En primer lugar, se elige el idioma y el teclado a utilizar. En este caso se ha utilizado la imagen (iso) de Windows Server 2019 en inglés: English (United States) y el teclado en español: Spanish (Spain, International Sort) y spanish como método de entrada.

\begin{figure}[H] %[H] para here [b] para bottom [t] para top
\begin{center}
\includegraphics[width=10cm]{DC01/Instalacion1.png}
\end{center}
\end{figure}

\item Se empieza con la instalación pulsando en ``Install Now''.

\begin{figure}[H] %[H] para here [b] para bottom [t] para top
\begin{center}
\includegraphics[width=10cm]{DC01/Instalacion2.png}
\end{center}
\end{figure}


\item Test

\begin{figure}[H] %[H] para here [b] para bottom [t] para top
\begin{center}
\includegraphics[width=10cm]{DC01/Instalacion3.png}
\end{center}
\end{figure}

\item Test

\begin{figure}[H] %[H] para here [b] para bottom [t] para top
\begin{center}
\includegraphics[width=10cm]{DC01/Instalacion4.png}
\end{center}
\end{figure}

\item Test

\begin{figure}[H] %[H] para here [b] para bottom [t] para top
\begin{center}
\includegraphics[width=10cm]{DC01/Instalacion5.png}
\end{center}
\end{figure}

\item Test

\begin{figure}[H] %[H] para here [b] para bottom [t] para top
\begin{center}
\includegraphics[width=10cm]{DC01/Instalacion6.png}
\end{center}
\end{figure}

\item Test

\begin{figure}[H] %[H] para here [b] para bottom [t] para top
\begin{center}
\includegraphics[width=10cm]{DC01/Instalacion7.png}
\end{center}
\end{figure}

\item Test

\begin{figure}[H] %[H] para here [b] para bottom [t] para top
\begin{center}
\includegraphics[width=10cm]{DC01/Instalacion8.png}
\end{center}
\end{figure}

\end{enumerate}


\textbf{Actualización}

\begin{enumerate}
\item Test

\begin{figure}[H] %[H] para here [b] para bottom [t] para top
\begin{center}
\includegraphics[width=10cm]{DC01/Update1.png}
\end{center}
\end{figure}

\item Test

\begin{figure}[H] %[H] para here [b] para bottom [t] para top
\begin{center}
\includegraphics[width=10cm]{DC01/Update2.png}
\end{center}
\end{figure}

\end{enumerate}

\textbf{Configuración}



%
%\chapter*{Instalación Windows 10 Enterprise (Cliente01)}
%\addcontentsline{toc}{chapter}{Anexo 2: Instalación Windows 10 Enterprise (Cliente01)}
%\textbf{Creación de Máquina Virtual}
\begin{enumerate}

\item En primer lugar, elegimos el nombre de la máquina: \textbf{Cliente01} y el tipo, en este caso se trata de \textbf{Microsoft Windows} versión \textbf{Microsoft Windows 10 (64-bit)}.
\begin{figure}[H] %[H] para here [b] para bottom [t] para top
\begin{center}
\includegraphics[width=10cm]{Cliente01/MV1.png}
\end{center}
\end{figure}

\item En el segundo paso, elegimos la RAM que vamos a destinar a la máquina virtual, aunque el requisito mínimo es de 2GB vamos a destinar 4GB: \textbf{4096 MB} para mejorar la experiencia a la hora de usar esta máquina.
\begin{figure}[H] %[H] para here [b] para bottom [t] para top
\begin{center}
\includegraphics[width=10cm]{Cliente01/MV2.png}
\end{center}
\end{figure}

\item En esta opción, se elige \textbf{Crear un disco duro virtual ahora}.
\begin{figure}[H] %[H] para here [b] para bottom [t] para top
\begin{center}
\includegraphics[width=10cm]{Cliente01/MV3.png}
\end{center}
\end{figure}

\item En cuanto al tipo de disco duro virtual se elige \textbf{VDI (Virtualbox Disk Image)}.
\begin{figure}[H] %[H] para here [b] para bottom [t] para top
\begin{center}
\includegraphics[width=10cm]{Cliente01/MV4.png}
\end{center}
\end{figure}

\item Por último, se ha elegido \textbf{30 GBs} de espacio de disco duro virtual.
\begin{figure}[H] %[H] para here [b] para bottom [t] para top
\begin{center}
\includegraphics[width=10cm]{Cliente01/MV5.png}
\end{center}
\end{figure}

\item Para arrancar la imagen del sistema operativo, hay que seleccionarla desde Con\-fi\-gu\-ra\-ción/\-Almacenamiento de la máquina virtual creada como Unidad Óptica.
\begin{figure}[H] %[H] para here [b] para bottom [t] para top
\begin{center}
\includegraphics[width=10cm]{Cliente01/MV6.png}
\end{center}
\end{figure}




\end{enumerate}
\textbf{Instalación}

\textbf{Actualización}

%
%
%\chapter*{Instalación Debian 10 (Gateway)}
%\addcontentsline{toc}{chapter}{Anexo 3: Instalación Debian 10 (Gateway)}
%\textbf{Creación de Máquina Virtual}
\begin{enumerate}

\item En primer lugar, elegimos el nombre de la máquina: \textbf{Gateway} y el tipo, en este caso se trata de \textbf{Linux} versión \textbf{Debian (64-bit)}.
\begin{figure}[H] %[H] para here [b] para bottom [t] para top
\begin{center}
\includegraphics[width=10cm]{Gateway/MV1.png}
\end{center}
\end{figure}

\item En el segundo paso, elegimos la RAM que vamos a destinar a la máquina virtual, al tratarse de un Debian sin entorno gráfico no es necesario destinarle muchos requisos por lo que elegimos 1GB: \textbf{1024 MB}.
\begin{figure}[H] %[H] para here [b] para bottom [t] para top
\begin{center}
\includegraphics[width=10cm]{Gateway/MV2.png}
\end{center}
\end{figure}

\item En esta opción, se elige \textbf{Crear un disco duro virtual ahora}.
\begin{figure}[H] %[H] para here [b] para bottom [t] para top
\begin{center}
\includegraphics[width=10cm]{Gateway/MV3.png}
\end{center}
\end{figure}

\item En cuanto al tipo de disco duro virtual se elige \textbf{VDI (Virtualbox Disk Image)}.
\begin{figure}[H] %[H] para here [b] para bottom [t] para top
\begin{center}
\includegraphics[width=10cm]{Gateway/MV4.png}
\end{center}
\end{figure}

\item Por último, se ha elegido \textbf{10 GBs} de espacio de disco duro virtual.
\begin{figure}[H] %[H] para here [b] para bottom [t] para top
\begin{center}
\includegraphics[width=10cm]{Gateway/MV5.png}
\end{center}
\end{figure}

\item Para arrancar la imagen del sistema operativo, hay que seleccionarla desde Con\-fi\-gu\-ra\-ción/\-Almacenamiento de la máquina virtual creada como Unidad Óptica.
\begin{figure}[H] %[H] para here [b] para bottom [t] para top
\begin{center}
\includegraphics[width=10cm]{Gateway/MV6.png}
\end{center}
\end{figure}




\end{enumerate}
\textbf{Instalación}

\textbf{Actualización}

%
%
%\chapter*{Instalación Kali Linux 2019.3 (Atacante01)}
%\addcontentsline{toc}{chapter}{Anexo 4: Instalación Kali Linux 2019.3 (Atacante01)}
%\textbf{Creación de Máquina Virtual}
\begin{enumerate}

\item En primer lugar, elegimos el nombre de la máquina: \textbf{Atacante01} y el tipo, en este caso se trata de \textbf{Linux} versión Kali Linux 2019.3, como esta opción no está disponible elegimos \textbf{Other Linux (64-bit)}.
\begin{figure}[H] %[H] para here [b] para bottom [t] para top
\begin{center}
\includegraphics[width=10cm]{Atacante01/MV1.png}
\end{center}
\end{figure}

\item En el segundo paso, elegimos la RAM que vamos a destinar a la máquina virtual, al tratarse de la máquina donde se van a realizar las pruebas se destinan 4GB: \textbf{4096 MB}.
\begin{figure}[H] %[H] para here [b] para bottom [t] para top
\begin{center}
\includegraphics[width=10cm]{Atacante01/MV2.png}
\end{center}
\end{figure}

\item En esta opción, se elige \textbf{Crear un disco duro virtual ahora}.
\begin{figure}[H] %[H] para here [b] para bottom [t] para top
\begin{center}
\includegraphics[width=10cm]{Atacante01/MV3.png}
\end{center}
\end{figure}

\item En cuanto al tipo de disco duro virtual se elige \textbf{VDI (Virtualbox Disk Image)}.
\begin{figure}[H] %[H] para here [b] para bottom [t] para top
\begin{center}
\includegraphics[width=10cm]{Atacante01/MV4.png}
\end{center}
\end{figure}

\item Por último, se ha elegido \textbf{15 GBs} de espacio de disco duro virtual.
\begin{figure}[H] %[H] para here [b] para bottom [t] para top
\begin{center}
\includegraphics[width=10cm]{Atacante01/MV5.png}
\end{center}
\end{figure}

\item Para arrancar la imagen del sistema operativo, hay que seleccionarla desde Con\-fi\-gu\-ra\-ción/\-Almacenamiento de la máquina virtual creada como Unidad Óptica.
\begin{figure}[H] %[H] para here [b] para bottom [t] para top
\begin{center}
\includegraphics[width=10cm]{Atacante01/MV6.png}
\end{center}
\end{figure}




\end{enumerate}
\textbf{Instalación}

\textbf{Actualización}




\end{document}
